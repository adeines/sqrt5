\newcommand{\n}{\mathfrak{n}}
\newcommand{\ap}[1]{a_{\p_{#1}}}
\newcommand{\round}[1]{\left\lfloor{#1}\right\rceil}
\newcommand{\fc}{\mathfrak{c}}

In (INSERT REFERENCE), Lassina Dembele outlines a method for finding modular elliptic
curves from Hilbert modular forms over totally real fields. In the case of nonsquare level
the method relies on computing or guessing periods of the curve from L-functions of twists
of the curve evaluated at 1. In particular, the only inputs required are the level of the
Hilbert modular form and its L-series. So we suppose that we know the level $\n = (N)$ of the form,
where $N$ is totally positive,
and that we have sufficiently many coefficients of its L-series $\ap{1}, \ap{2}, \ap{3}, \ldots$.

Associated to an elliptic curve is a period lattice \ldots (explain something here?)

Suppose $E$ is an elliptic curve over $\Q(\sqrt 5)$. It does not make sense to speak of
\emph{the} period lattice of $E$, as there are multiple embeddings of $E$ into the complex
numbers. So let $\alpha_1$ and $\alpha_2$ denote the embeddings of $K$ into the real numbers,
for concreteness with $\alpha_1$ chosen so that $\alpha_1(\phi) > 0$. Then we get two period
lattices $L_{\alpha_1}(E)$ and $L_{\alpha 2}(E)$, corresponding to the two different embeddings
of $K$ in $\R$. For convenience, we will henceforth choose $\alpha_1$ as a distinguished embedding;
we think of $L_{\alpha_1}(E)$ as the period lattice of $E$ and $L_{\alpha_2}(E)$ as the period lattice
of $\bar E$. (Note that $L_{\alpha_1}(\bar E) = L_{\alpha_2}(E)$.)

\newcommand{\Omegap}{\Omega^+}
\newcommand{\Omegam}{\Omega^-}
\newcommand{\Omegapp}{\Omega^{++}}
\newcommand{\Omegapm}{\Omega^{+-}}
\newcommand{\Omegamp}{\Omega^{-+}}
\newcommand{\Omegamm}{\Omega^{--}}

So, with this understanding, let $\Omegap_E$ denote the least real period of $E$, and let
$\Omegam_E$ denote the least imaginary period of $E$, a purely imaginary number; thus, we either
have $L_{\alpha_1}(E) = \Omegap\Z + \Omegam\Z$ or $L_{\alpha_1}(E) = \Omegap\Z + 1/2(\Omegap + \Omegam)\Z$.
Furthermore,
as periods will often occur mixed together in pairs, we define
\begin{eqnarray*}
\Omegapp_E &=& \Omegap_E\Omegap_{\bar E} \\
\Omegapm_E &=& \Omegap_E\Omegam_{\bar E} \\
\Omegamp_E &=& \Omegam_E\Omegap_{\bar E} \\
\Omegamm_E &=& \Omegam_E\Omegam_{\bar E}.
\end{eqnarray*}
We will refer to these numbers as the mixed periods $E$.

Dembele's method relies on two observations. First, if we know these four numbers (or actually just the first three,
but from any three of them we can recover the fourth), we can recover the curve
$E$, and second, we can compute these numbers, or at least integer multiples of these numbers, by computing
central values of twists of the L-function associated to $E$.

\subsubsection{Recovering the curve from its mixed periods}

First, let us assume that we know the mixed periods of the curve to infinite precision. Absent the knowledge
of the discriminant of the curve, we do not precisely know the the lattice type of the curve and its conjugate,
but there are only a few possibilities for what they may be. So we compute the $4$ complex numbers
\begin{eqnarray*}
    \tau_1(E) &=& \frac{\Omegamp_E}{\Omegapp_E} = \frac{\Omegam_E}{\Omegap_E} \\
    \tau_2(E) &=& \frac{1}{2}\left(1 + \frac{\Omegamp_E}{\Omegapp_E}\right) = \frac{1}{2}\left(1 + \frac{\Omegam_E}{\Omegap_E}\right) \\
    \tau_1(\bar E) &=& \frac{\Omegapm_E}{\Omegapp_E} = \frac{\Omegam_E}{\Omegap_E} \\
    \tau_2(\bar E) &=& \frac{1}{2}\left(1 + \frac{\Omegapm_E}{\Omegapp_E}\right) = \frac{1}{2}\left(1 + \frac{\Omegam_{\bar E}}{\Omegap_{\bar E}}\right).
\end{eqnarray*}
We now compute $j(\tau)$ for each of numbers above, where (letting $q = \exp(2 \pi i \tau)$)
\[
    j(\tau) = \frac{1}{q} + 744 + 196884q + 21493760q^{2} + \cdots
\]
is the modular $j$-function. (In practice, we want to normalize $\tau$ to be in the standard fundamental
domain for the action of $SL(2, \Z)$, so that the Fourier series for $j(\tau)$ converges nicely.) This should
give us $4$ real numbers $j_1(E), j_2(E), j_1(\bar E),$ and $j_2(\bar E)$, and we will have
$\sigma_1(j(E)) = j_{(1 \textrm{ or } 2)}(E)$ and $\sigma_2(j(E)) = j_{(1 \textrm{ or } 2)}(\bar E)$. We try
each of the four possibilities in turn, recognizing $j(E)$ as an algebraic number from its two embeddings and
constructing a curve with the desired $j$-invariant. Once we have a curve we can examine its Fourier coefficients
to determine whether or not it is a twist of the curve we are looking for, and, if it is, we determine
which twist it is.

In practice, of course, we only have limited precision, and as $j(E)$ will not be an algebraic integer it may
not be feasible to directly determine it exactly, especially if it has a rather large denominator. Still,
the above approach may be useful.

To get around issues of limited precision, we suppose that we have some extra information; namely, the discriminant
$\Delta_E$ of the curve we are looking for. (In practice we start by guessing that $\Delta_E = N$, and then we
introduce unit factors and increment powers of the factors of $N$ until we find the curve that we are looking for.)
With $\Delta_E$ in hand we can directly determine which $\tau$ to choose: if $\sigma_1(\Delta_E) > 0$ then
$\sigma_1(j(E)) = j(\tau_1(E))$, and if $\sigma_1(\Delta_E) < 0$ then $\sigma_1(j(E)) = j(\tau_2(E))$, and
similarly for $\sigma_2$. We now compute $\sigma_1(c_4(E)) = (j(\tau) \sigma_1(\Delta_E))^{1/3}$ and
$\sigma_2(c_4(E)) = (j(\tau') \sigma_2(\Delta_E))^{1/3}$.

With approximations of the two embeddings of $c_4$ in hand we can recognize $c_4$ approximately as an algebraic
integer. Specifically, we compute
\[
    \alpha = \frac{1}{2}\round{\sigma_1(c_4) + \sigma_2(c_4)}
\]
and
\[
    \beta = \frac{1}{2}\round{\frac{\sigma_1(c_4) - \sigma_2(c_4)}{\sqrt{5}}},
\]
where $\round(x)$ denotes the nearest integer function. We then set $C' = \alpha + \beta\sqrt{5}$. This may
not actually be an integer, so we convert it to $C = a + b \varphi$, arbitrarily rounding either $a$ or $b$
in the case that $C'$ was not actually integer. Now we vary $m$ and $n$ in some range around $0$ and for
each of the possibilities
\[
    c_{4, \mathrm{guess}} = (a + m) + (b + n)\phi
\]
we attempt to solve
\[
    c_{6, \mathrm{guess}} = \pm \sqrt{c_{4, \mathrm{guess}}^3 - 1728 \Delta_E}.
\]
Each time this has a solution we construct a curve $E_{\mathrm{guess}}$ with given $c_4$ and $c_6$, check
if it has the correct conductor, and, if so, check if its Fourier coefficients are the ones that we are looking
for. If it passes these checks, we declare that $E = E_{\mathrm{guess}}$.

\subsubsection{Dirichlet characters and twists}

To compute the mixed periods of the curve we will need to compute central values of the $L$-function
twisted by quadratic Dirichlet characters over $O_K$. Specifically, such a character $\chi$ is a completely
multiplicative function $O_K \rightarrow {-1, 0, 1}$ that is periodic with respect to some minimal ideal $\a$,
so that $\chi(x)$ = $\chi(n + a)$ for all $n \in O_K$ and all $a \in \a$. Equivalently, we may consider a
homomorphism $\chi : (O_K/\a)^\cross \rightarrow {\pm 1}$ and extend it to $O_K$ by lifting and defining
$\chi(n)$ to be $0$ whenever $x$ is not relatively prime to $\a$.

For any $\a$ there is a \emph{trivial} character $\chi_0(n)$, which is the function that takes the value $1$
whenever $n$ is relatively prime to $\a$ and $0$ otherwise. We say a character $\chi$ is imprimitive if its
least period is $\a$ and it can be written as $\chi(n) = \chi_0(n)\chi_1(n)$, where $\chi_1(n)$ is a character
with a smaller period than $\chi$. If $\chi_1$ has smallest possible period for
such a representation, we call the period of $\chi_1$ the conductor of $\chi$. If the conductor of $\chi$ is
also the modulus of $\chi$, then we say that $\chi$ is primitive. These are the characters that we are
interested in.

For simplicity, we compute only with characters of prime modulus. For each prime modulus $\p$ there is
exactly one quadratic character, and it is primitive. To construct it and compute with it, one can simply
construct $O_K/\p$, which will be a finite field, and will thus have a multiplicative generator. We choose
such a generator and assign it the value $-1$; this completely determines the values of $\chi$. We can
construct a table of these values and use it for evaluation of $\chi$.

The use of $\chi$ comes in twisting the $L$-function of $E$ by $\chi$ to compute central values
which are related to the mixed periods of $E$. If we have
\[
    L(E, s) = \sum_{\m \subseteq \cO_K} \frac{a_\m}{N(\m)^s},
\]
then for primitive $\chi$ with conductor relatively prime to the conductor of $E$, the twisted $L$-function
is given by
\[
    L(E, \chi, s) = \sum_{\m \subseteq \cO_K} \frac{\chi(m) a_\m}{N(\m)^s},
\]
where $m$ is a totally positive generator of $\m$. The functional equation satisfied by $L(E, \chi, s)$ is
the same as that of $L(E, s)$ except the conductor is multiplied by the square of the norm of the conductor
of $\chi$ and the sign is multiplied by $\chi(-N)$.

\subsubsection{Finding the mixed periods}
The key to finding the periods of $E$ is the following conjecture of SOMEONE. (Is this a natural generalization
of something that is known over $\Q$?)
\begin{conjecture} If $\chi$ is primitive with conductor $\fc$ relatively prime to the conductor of $E$,
with $\chi(\varphi) = s'$ and $\chi(1 - \varphi) = s$, $s, s' \in \{+, -\} = \{+1, -1\}$, then
\[
    \Omega^{s,s'}_E = c_\chi i^{ss'} \sqrt{5} \sqrt{(2, N(\fc))N(\fc)} L(E, \chi, 1)
\]
for some integer $c_\chi$.
\end{conjecture}
\begin{remark}
I've specialized the conjecture to $\Q(\sqrt 5)$ and I've slightly modified the it to include what I believe to be
the absolute value of the Gauss sum, and its sign up to $\pm 1$. Perhaps I need to look at this a little bit more
carefully.
\end{remark}

With this conjecture in place we can now fairly easily compute integer multiples of each of the mixed periods of
$E$ when the conductor is not a square, and with some extra work sometimes we even recover the periods directly.
When the conductor is a square, there is a problem, however. The sign of $L(E, \chi, s)$ is given by
$\epsilon_E \chi(-N),$ which is the same as $\epsilon_E \chi(-1)$ when $N$ is a square. $\chi(-1)$ is determined
by $\chi(-1) = \chi(\varphi)\chi(1 - \varphi)$, so if $\epsilon_E = 1$, then $L(E, \chi, 1)$ will only be 
nonvanishing when $\chi(\varphi) = \chi(1 - \varphi)$, and we can only obtain information about $\Omegapp$ and
$\Omegamm$. Similarly, when $\epsilon_E = -1$ and $N$ is a square we can only obtain information about
$\Omegapm$ and $\Omegamp$.

Suppose then that $N$ is not a square, and again assume that we can compute with infinite precision, or
at least all the precision we like. Then to find the mixed periods we choose some bound $M$ on the conductors
of the characters we are willing to consider and first make four lists of characters
\[
    S^{s,s'} = \{ \chi \bmod \p : \chi(\phi) = s', \chi(1 - \phi) = s, (\p, \n) = 1, N(\p) < M, \chi(-N) = \epsilon_E\}
\]
Here $s, s' \in \{\pm 1\} = \{+, -\}$. We will consider these lists to be ordered by the norm conductors of the
characters in increasing order, and index their elements as $\chi^{s,s'}_0, \chi^{s,s'}_1, \chi^{s,s'}_2, \ldots$.
For each character we compute the central value of the twisted $L$-function to get four new lists
\[
    \mathcal{L}^{s,s'} = \{ i^{ss'}\sqrt{5 (2, N(\p)), N(\p)} L(E,\chi,1), \chi \in S^{s,s'}\} =
        \{\mathcal{L}^{s,s'}_0, \mathcal{L}^{s,s'}_1, \ldots\}
\]
These should all be integer multiples of the mixed periods that we are looking for. To get a good idea of which
multiples we should guess they are, we compute each of the ratios
\[
    \frac{\mathcal{L}^{s,s'}_0}{\mathcal{L}^{s,s'}_k} = \frac{c_{\chi^{s,s'}_0}}{c_{\chi^{s,s'}_k}} \in \Q, k = 1, 2, \ldots
\]
and recognize these as rational numbers (ignoring precision issues for the moment). If it turns out that
any pair $c_{\chi^{s,s'}_k},c_{\chi^{s,s'}_l}$ is relatively prime, will be able to determine $\Omega^{ss'}$
exactly. Regardless of this, we choose as an initial guess
\[
    \Omega^{ss'}_{E, \mathrm{guess}} = \frac{\mathcal{L}^{s,s'}_0}{M^{s,s'}}
\]
where
\[
    M^{s,s'} = \mathrm{lcm}\left\{ \mathrm{numerator}\left(\frac{\mathcal{L}^{s,s'}_0}{\mathcal{L}^{s,s'}_k}\right),
        k = 1,2, \ldots \right\}
\]

\subsubsection{Examples and putting things together}
The above discussion leaves some issues to be determined. In pratice we have only a limited number of coefficients
of the $L$-functions that we need to work with, so our precision is somewhat limited. Also, there are a number
of guesses that need to be made in the algorithm, so we need to have a good way of understanding how to make those
guesses. We start with a few examples, which may provide some enlightenment, and then proceed to make
suggestions for how a general search method should work.

\textbf{Example 1.} We start with a simple example, using the same curve a Dembele uses in his Example 1. So let
$E$ be the curve over $K = \Q( \sqrt 5)$ with $a$-invariants $[1, -(1+\varphi), \varphi, 0, 0]$, so
\[
    E := y^2 + xy + \varphi y = x^3 - (\varphi+1)x^2.
\]
Using Sage, we compute a basis for the period lattice of E (recall that we are setting a 
distinguished embedding) and find one that is (approximately)
\[
(3.05217315335726, 2.39884476932372i).
\]
With the other embedding of $K$ into $\RR$ we find a basis for the period lattice of $\bar E$ as
\[
(8.43805988789973, 4.21902994394986 + 1.57216678613265i).
\]
So we have
\begin{eqnarray*}
    \Omegapp_E &\approx& 3.05217315335726 \times 8.43805988789973 \approx 25.7544198562683 \\
    \Omegapm_E &\approx& 3.05217315335726 \times (2 \cdot 1.57216678613265 i) \approx 9.59705051446808i \\
    \Omegamp_E &\approx& 2.39884476932372i \times 8.43805988789973 \approx 20.2415958253286i \\
    \Omegamm_E &\approx& 2.39884476932372i \times (2 \cdot 1.57216678613265 i) \approx -7.54276814283759
\end{eqnarray*}
Note that here we have $c_4(E) = 16 \varphi + 25$ and $\Delta_E = 16 \varphi + 9$, from which we can compute
$j(E) = c_4(E)^3/\Delta_E = \frac{1}{31}(106208\varphi - 54753)$, and find that
\[
    \sigma_1(j(E)) \approx 3777.26302829512, \ \ \  \sigma_2(j(E)) \approx -3883.65012506932.
\]
We can also compute these numbers from the mixed periods of $E$. Since $\sigma_1(\Delta) > 0$, we have
\[
    \sigma_1(j(E)) = j(\Omegamp_E/\Omegapp_E) \approx j(0.785946487565785i)
\]
If we compute this directly using $10$ terms in the Fourier expansion of the $j$-function, we will find that
\[
    j(0.785946487565785i) \approx 3777.26301983227,
\]
which is not too far from the actual $j$-invariant. We can do better however. Under the action of $\SL(2, \Z)$,
$0.785946487565785i$ maps to $1.27235125523263i$, and if we compute the $j$-function here using ten terms
we will find that
\[
    j(1.27235125523263i) \approx 3777.26302829512,
\]
which completely agrees with the $j$-invariant computed above. Similarly, since $\sigma_2(\Delta) < 0$,
we can compute that
\[
    \sigma_2(j(E)) = j\left( \frac{1}{2}\left(1 + \Omegamm_E/\Omegapm_E\right)\right)
            \approx j(0.5 + 0.186318514803048i).
\]
In this case if we try to compute the with ten terms in the Fourier expansion at $53$ bits of precision
we will find that
\[
     j(0.5 + 0.186318514803048i) \approx -5.93248167367531\cdot 10^{10} + 0.0000675970580457112i,
\]
which is complete nonsense. We can do better by using something like $200$ terms and $200$ bits of precision, which will give
\[
    j(0.5 + 0.186318514803048i) \approx -3883.6501250693112
\]
or we just translate $0.5 + 0.186318514803048i$ to $0.5 + 1.34178828263132$, use ten terms at $53$ bits of
precision, and find that
\[
    j(0.5 + 1.34178828263132) \approx -3883.65012506932.
\]
With this much precision for the two embeddings of $j(E)$, it is somewhat possible to recover $j(E)$ as
an algebraic number, though we really only would have such confidence because we already know what $j(E)$ is.
Set $j_1 = \sigma_1(j(E))$ and $j_2 = \sigma_2(j(E))$. If we compute the continued fraction expansion of
$.5(j_1 + j_2)$, for example we will get the sequence of convergents
\[
    -54, -53, -\frac{266}{5}, -\frac{1383}{26}, -\frac{1649}{31},
\]
and the last one is in fact exactly $2\tr(j(E))$ However, it seems recognizing $j(E)$ directly this
way would be fraught with precision issues. Instead we can use our knowledge of $\Delta_E$ to work
backwards, and find that
\[
    \sigma_1(c_4) = \sigma_1(\Delta_E) \approx 50.8885438199983
\]
and
\[
    \sigma_2(c_4) = j_2 \sigma_2(\Delta_E) \approx 15.1114561800017.
\]
With these we can easily recognize $c_4$ as an algebraic integer. We compute
\[
    \tr(c_4) = 66
\]
and
\[
    \tr(c_4/\sqrt{5}) = (\sigma_1(c4) - \sigma_2(c4))/\sigma_1(\sqrt{5}) = 16,
\]
so $c_4 = 33 + 8\sqrt{5} = 33 + 8(2 \varphi - 1) = 25 + 16 \varphi$. We solve for
$c_6 = \pm \sqrt(1728\Delta_E - c_4^3)$. There are two possibilities for $c_6$, so we try each
of the two curves, and find that one of them is the curve that we are looking for.

We still have not dealt with the issue of how we might find the periods of the curve without knowing
the curve. In this case the functional equation for $L(E, s)$ has sign $+1$, so we will twist by characters
$\chi$ such that $\chi(-N) = 1$. So we look through the prime ideals of $O_K$ in order of norm
and we find that the characters with conductors $(3), (3 \varphi - 1), (3 \varphi - 2)$ and $(\varphi + 5)$
suit our purposes. With these choices, we have
\begin{eqnarray*}
    \chi_{(3)}(\phi) = -1, \ \  && \ \ \chi_{(3)}(1 - \varphi) = -1 \\
    \chi_{(3\varphi - 1)}(\phi) = 1, \ \  && \ \ \chi_{(3\varphi - 1)}(1 - \varphi) = -1 \\
    \chi_{(3\varphi - 2)}(\phi) = -1, \ \  && \ \ \chi_{(3\varphi - 2)}(1 - \varphi) = 1 \\
    \chi_{(\varphi + 1)}(\phi) = 1, \ \  && \ \ \chi_{(\varphi + 2)}(1 - \varphi) = 1.
\end{eqnarray*}
Thus, for example, $L(E, \chi_{(3)}, 1) \sqrt{5 N(3)}$ should be a multiple of $\Omegamm_E$. Indeed, using all
of the primes of norm up to $40000$ we compute $ L(E, \chi_{(3)}, 1) \approx 1.12440948706034$, and
$3 \sqrt {5} \times 1.12440948706034 \approx 7.54276814283777$, which we recognize as very close to
$\Omegamm_E$.

\textbf{Example 2.} We now proceed to a more complicated example. Suppose that we have found $a_\p$ for all
primes of norm less than $40000$ for a Hilbert modular form over $K$ of level $30*a - 10$, with $\epsilon_f = 1$.
We examine the characters with conductors up to $1500$ and taking at most $10$ of each type that we
are looking for, we build the lists

\begin{multline}
S^{--} = \big\{\chi_{(a + 6)}, \chi_{(7)}, \chi_{(7a - 3)}, \chi_{(a + 10)}, \chi_{(-11a + 7)}, \\
\chi_{(-14a + 5)}, \chi_{(-15a + 4)}, \chi_{(-15a + 11)}, \chi_{(15a - 8)}, \chi_{(-18a + 13)}\big\}
\end{multline}
\begin{multline}
S^{-+} = \big\{\chi_{(a - 9)}, \chi_{(-8a + 5)}, \chi_{(a - 12)}, \chi_{(2a + 11)}, \chi_{(-12a + 5)}, \\
\chi_{(3a + 13)}, \chi_{(13a - 6)}, \chi_{(a - 16)}, \chi_{(16a - 5)}, \chi_{(-3a - 17)}\big\}
\end{multline}
\begin{multline}
S^{+-} = \big\{\chi_{(5a - 3)}, \chi_{(7a - 2)}, \chi_{(-8a + 3)}, \chi_{(a + 11)}, \chi_{(-2a + 13)}, \\
\chi_{(2a + 13)}, \chi_{(13a - 7)}, \chi_{(a + 15)}, \chi_{(-2a + 17)}, \chi_{(-17a + 7)}\big\}
\end{multline}
\begin{multline}
S^{++} = \big\{\chi_{(-a + 6)}, \chi_{(9a - 4)}, \chi_{(a - 14)}, \chi_{(a + 13)}, \chi_{(-17a + 5)}, \\
\chi_{(a - 22)}, \chi_{(-3a + 25)}, \chi_{(4a + 25)}, \chi_{(-4a + 29)}, \chi_{(25a - 8)}\big\}
\end{multline}

For each of these characters we evaluate $L(f, \chi, 1)$ using lcalc, and multiply by the appropriate
factors to get the lists $\mathcal{L}^{s,s'}$ above. Note that lcalc will warn us
that we don't have enough coefficients for accurate evaluation. This is correct, and in particular the
accuracy of the numbers in the following lists drops off as we more futher along the lists. So we
get the following lists of approximate values.

\begin{multline}
\mathcal{L}^{--} = \big\{23.0447957401108,
23.0447957401090,
69.1343872203071,
\\69.1343870414420,
276.537547685983,
0,
69.1339752521539,
\\69.1347565067475,
92.1785740140933,
69.1381805217699\big\}
\end{multline}
\begin{multline}
\mathcal{L}^{-+} = \big\{13.8729392831410i,
13.8729392838598i,
115.607826237882i,
\\226.591340602765i,
41.6188067753373i,
221.967022543403i,
226.591333886957i,
\\0,
124.855687410063i,
226.589793508735i\big\}
\end{multline}
\begin{multline}
\mathcal{L}^{+-} = \big\{36.8547445634673i,
12.2849148547115i,
36.8547445497484i,
\\12.2849138747566i,
196.558637741332i,
147.418978438534i,
307.122767748292i,
\\36.8546922087215i,
110.564250431060i,
0\big\}
\end{multline}
\begin{multline}
\mathcal{L}^{++} = \big\{66.5595355493066,
66.5595355094869,
0,
\\66.5595717095333,
0,
599.042164153531,
266.230056409565,
\\266.076837348988,
66.5736728191717,
61.5776839596031\big\}
\end{multline}

Recall now that these numbers are all supposed to be integer multiples of the mixed periods of the
curve. The first number in each list is likely to be the most accurate, so we divide each list by the
first number and get a list of what should be rational numbers. We will use these to determine a good guess
for which integer to divide the first entry by.

\begin{multline}
 = \big\{1.00000000000000,
1.00000000000008,
0.333333333333455,
\\0.333333334195858,
0.0833333336935456,
0.333335319660977,
0.333331552818323,
\\0.250001651539841,
0.333315044830469\big\}
\end{multline}
\begin{multline}
 = \big\{1.00000000000000,
0.999999999948185,
0.120000001164239,
\\0.0612244900720257,
0.333333422027897,
0.0625000016857383,
0.0612244918866227,
\\0.111111792909987,
0.0612249080963401\big\}
\end{multline}
\begin{multline}
 = \big\{1.00000000000000,
2.99999999994569,
1.00000000037224,
\\3.00000023925259,
0.187499999933697,
0.249999999686837,
0.120000040484372,
\\1.00000142057205,
0.333333282862957\big\}
\end{multline}
\begin{multline}
 = \big\{1.00000000000000,
1.00000000059826,
0.999999456723869,
\\0.111109934378923,
0.250007592857631,
0.250151558521446,
0.999787644736028,
\\1.08090352331166\big\}
\end{multline}

Many of these are immediately recognizable as approximate rational numbers, while others take a little more effort.
A hueristic attempt to recognize each one automatically by examining the continued fraction convergents
and looking for big jumps in the denominators is likely to come up with lists of rational numbers that
look like the following.

\begin{equation*}
\big\{1,
1,
\frac{1}{3},
\frac{1}{3},
\frac{1}{12},
\frac{1}{3},
\frac{1}{3},
\frac{1}{4},
\frac{1}{3}\big\}
\end{equation*}
\begin{equation*}
\big\{1,
1,
\frac{3}{25},
\frac{3}{49},
\frac{1}{3},
\frac{1}{16},
\frac{3}{49},
\frac{1}{9},
\frac{3}{49}\big\}
\end{equation*}
\begin{equation*}
\big\{1,
3,
1,
3,
\frac{3}{16},
\frac{1}{4},
\frac{3}{25},
1,
\frac{1}{3}\big\}
\end{equation*}
\begin{equation*}
\big\{1,
1,
1,
\frac{1}{9},
\frac{1}{4},
\frac{1}{4},
1,
\frac{8632075}{7985981}\big\}
\end{equation*}
These lists look reasonable except for the last entry in the final list. To examine it more closely,
we look at the convergents of the continued fraction expansion of $1.08090352331166$, which look like
\begin{multline}
\big[1, \frac{13}{12}, \frac{27}{25}, \frac{40}{37}, \frac{147}{136}, \frac{334}{309}, \frac{1149}{1063},
\frac{1483}{1372}, \frac{2632}{2435}, \frac{12011}{11112}, \frac{14643}{13547}, \frac{26654}{24659},
\frac{41297}{38206}, \\ 
\frac{67951}{62865}, \frac{109248}{101071}, \frac{177199}{163936}, \frac{286447}{265007}, \frac{463646}{428943},
\frac{8168429}{7557038}, \frac{8632075}{7985981}, \\ \frac{34064654}{31514981}, \frac{42696729}{39500962},
\frac{162154841}{150017867}\big].
\end{multline}
From this we can get an idea of what went wrong. We do not have enough precision for the denominator of
the continued fraction expansion to jump significantly enough for us to correctly guess the correct fraction.
The correct fraction is likely to be one of the first few. We have
\[
    13/12 = 1.08333\ldots,
\]
\[
    27/25 = 1.08
\]
\[
    40/37 = 1.081081\ldots,
\]
all of which are quite close to the value we are looking for. We choose $27/25$ as our guess for the correct
fraction on the assumption that small primes and squares are most likely to appear as factors. (This is
a somewhat weak justification, and difficult to automate.)

Examining these denominators, we make the guesses
\begin{eqnarray*}
    \Omegamm_{E, \textrm{guess}} = \mathcal{L}^{--}_0 / 1 &=& 23.0447957401108/1 = -23.0447957401108 \\
    \Omegamp_{E, \textrm{guess}} = \mathcal{L}^{-+}_0 / 3 &=& 13.8729392831410i/3 = 4.62431309438033i \\
    \Omegapm_{E, \textrm{guess}} = \mathcal{L}^{+-}_0 / 3 &=& 36.8547445634673i/3 = 12.2849148544891i \\
    \Omegapp_{E, \textrm{guess}} = \mathcal{L}^{++}_0 / 27 &=& 66.5595355493066/27 = 2.46516798330765.
\end{eqnarray*}
The actual periods will satisfy
\[
    \frac{\Omegamm_E \Omegapp_E}{\Omegamp_E \Omegapm_E} = 1,
\]
while in our case we have
\[
    \frac{\Omegamm_{E,\text{guess}} \Omegapp_{E, \text{guess}}}{\Omegamp_{E, \text{guess}} \Omegapm_{E,\text{guess}}} = 1.00000000027151,
\]
which is an indication that our guess is reasonable. (This relation could also be used to recognize that we
should choose $27/25$ above, and is useful in making guesses in general.) With these we get a few possibilities
for $\tau_E$ and $\tau_{\overline E}$. Translating these into the fundamental domain we get
\begin{eqnarray}
    t_1(E) &=& 1.87586124989975i \\
    t_2(E) &=& -0.5 + 0.937930624949874i \\
    t_1(\overline E) &=& 4.98339867208796i \\
    t_1(\overline E) &=& -0.5 + 2.49169933604398i
\end{eqnarray}
and $4$ corresponding pairs of guesses for the embeddings of $j(E)$:
\[
(132195.763704925, 3.967003831510192 \cdot 10^{13})
\]
\[
(132195.763704925, -6.297671571246 \cdot 10^{6} )
\]
\[
(-15.100427892327, 3.967003831510192 \cdot 10^{13}
\]
\[
(-15.100427892327, -6.297671571246 \cdot 10^{6} )
\]
(New idea:) It is hard to recognize any of these pairs as the embeddings of an algebraic number because the
denominators involved are quite large. However, we know something about the denominator; namely, $j(E)N^m$
is actually an algebraic integer for some power of $m$. So if we have enough precision, we should be able
to test a few powers of $N$ and find this integer. Indeed, with the fourth pair of possibilities, we find
that
\[
     -15.100427892327 \sigma_1(N)^3 -6297671.571246 \sigma_2(N)^3 \approx 146415110000
\]
and
\[
     -15.100427892327 \sigma_1(N)^3/\sqrt{5} + 629767.571246 \sigma_2(N)^3/\sqrt{5} \approx -65479601000,
\]
which suggests as a possibility for $j(E) N^3$
\[
    146415110000/2 - 65479601000\sqrt{5}/2 = -65479601000\varphi + 105947355500.
\]
This would give
\[
    j(E) = \frac{3748625197}{1331} \varphi - \frac{1213084615}{2662}.
\]
(This is in fact the correct $j$-invariant, and we should be able to construct a curve with this $j$
and recognize it as a twist of the curve we are looking for\ldots)
