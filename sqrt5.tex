\documentclass{amsart} 


% macros.tex
\usepackage{amsmath}
\usepackage{amsfonts}
\usepackage{amssymb}
\usepackage{amsthm}

\usepackage{url}


% You change everything, by adding \usepackage{times} to the document
% Preamble. Now all the roman letters will be set in times and all the
% sans serif stuff will be set in Helvetica. If you don't like times,
% you can try the packages: palatcm, charter, helvet, palatino, avant,
% newcent and bookman
% If you want to change explicitly to a certain font, use the command
% \fontfamily{XYZ}\selectfont whereby XYZ can be set to: pag for Adobe
% AvantGarde, pbk for Adobe Bookman, pcr for Adobe Courier, phv for
% Adobe Helvetica, pnc for Adobe NewCenturySchoolbook, ppl for Adobe
% Palatino, ptm for Adobe Times Roman, pzc for Adobe ZapfChancery
\newcommand{\courier}{\fontfamily{pcr}\selectfont}



\newcommand{\edit}[1]{\footnote{[[#1]]}\marginpar{\hfill {\sf[[\thefootnote]]}}}
%\newcommand{\edit}[1]{{\sl\small [[Todo: #1]]}}


%\author{William~A. Stein}

\newcommand{\Hbar}{\overline{H}}

\newcommand{\myhead}[3]{
\par\noindent
{Version #2}
\vspace{10ex}
\par\noindent
{\bf \LARGE #1}\\
\vspace{3ex}
\par\noindent
{\large W.\thinspace{}A. Stein}\\
{\small Department of Mathematics, Harvard University}\vspace{1ex}\\
#3     
\vspace{2ex}\par
}

\newcommand{\myheadauth}[3]{
\par\noindent
{Version #2}
\vspace{10ex}
\par\noindent
{\bf \LARGE #1}\\
\vspace{3ex}
\par\noindent
#3     
\vspace{5ex}\par
}

\usepackage{xspace}  % to allow for text macros that don't eat space 
\newcommand{\SAGE}{{\sf Sage}\xspace}
\newcommand{\sage}{\SAGE}
\newcommand{\gzero}{\Gamma_0(N)}
\newcommand{\esM}{\overline{\sM}}
\newcommand{\sM}{\boldsymbol{\mathcal{M}}}
\newcommand{\sS}{\boldsymbol{\mathcal{S}}}
\newcommand{\sB}{\boldsymbol{\mathcal{B}}}       
\newcommand{\bA}{\mathbb{A}}
\newcommand{\cK}{\mathcal{K}}
\newcommand{\Adual}{A^{\vee}}
\newcommand{\Bdual}{B^{\vee}}
\newcommand{\kr}[2]{\left(\frac{#1}{#2}\right)}

\newcommand{\defn}[1]{{\em #1}}
\newcommand{\solution}[1]{\vspace{1em}%
  \par\noindent{\bf Solution #1.} }
\newcommand{\todo}[1]{\noindent$\bullet$ {\small \textsf{#1}} $\bullet$\\}
\newcommand{\done}[1]{\noindent {\small \textsf{Done: #1}}\\}
\newcommand{\danger}[1]{\marginpar{\small \textsl{#1}}}
\renewcommand{\div}{\mbox{\rm div}}
\DeclareMathOperator{\GCD}{GCD}
\DeclareMathOperator{\CH}{CH}
\DeclareMathOperator{\sss}{ss}
\renewcommand{\ss}{\sss}
\DeclareMathOperator{\red}{red}
\DeclareMathOperator{\sat}{sat}
\DeclareMathOperator{\xgcd}{xgcd}
\DeclareMathOperator{\Kol}{Kol}
\DeclareMathOperator{\can}{can}
\DeclareMathOperator{\Cl}{Cl}
\DeclareMathOperator{\Mod}{Mod}
\DeclareMathOperator{\chr}{char}
\DeclareMathOperator{\charpoly}{charpoly}
\DeclareMathOperator{\cris}{cris}
\DeclareMathOperator{\dR}{dR}
\DeclareMathOperator{\Fil}{Fil}
\DeclareMathOperator{\ind}{ind}
\DeclareMathOperator{\im}{im}
\DeclareMathOperator{\oo}{\infty}
\DeclareMathOperator{\abs}{abs}
\DeclareMathOperator{\lcm}{lcm}
\DeclareMathOperator{\cores}{cores}
\DeclareMathOperator{\coker}{coker}
\DeclareMathOperator{\image}{image}
\DeclareMathOperator{\prt}{part}
\DeclareMathOperator{\proj}{proj}
\DeclareMathOperator{\Br}{Br}
\DeclareMathOperator{\Ann}{Ann}
\DeclareMathOperator{\End}{End}
\DeclareMathOperator{\Tan}{Tan}
\DeclareMathOperator{\Eis}{Eis}
\newcommand{\unity}{\mathbb{1}}
\DeclareMathOperator{\Pic}{Pic}
\DeclareMathOperator{\Tate}{Tate}
\DeclareMathOperator{\Vol}{Vol}
\DeclareMathOperator{\Vis}{Vis}
\DeclareMathOperator{\Reg}{Reg}
%\DeclareMathOperator{\myRes}{Res}
%\newcommand{\Res}{\myRes}
\DeclareMathOperator{\Res}{Res}
\newcommand{\an}{{\rm an}}
\DeclareMathOperator{\rank}{rank}
\DeclareMathOperator{\Sel}{Sel}
\DeclareMathOperator{\Mat}{Mat}
\DeclareMathOperator{\BSD}{BSD}
\DeclareMathOperator{\id}{id}
\DeclareMathOperator{\dz}{dz}
%\DeclareMathOperator{\Re}{Re}
\renewcommand{\Re}{\mbox{\rm Re}}
\DeclareMathOperator{\Imm}{Im}
\renewcommand{\Im}{\Imm}
\DeclareMathOperator{\Selmer}{Selmer}
\newcommand{\pfSel}{\widehat{\Sel}}
\newcommand{\qe}{\stackrel{\mbox{\tiny ?}}{=}}
\newcommand{\isog}{\simeq}
\newcommand{\e}{\mathbf{e}}
\newcommand{\bN}{\mathbf{N}}

% ---- SHA ----
\DeclareFontEncoding{OT2}{}{} % to enable usage of cyrillic fonts
  \newcommand{\textcyr}[1]{%
    {\fontencoding{OT2}\fontfamily{wncyr}\fontseries{m}\fontshape{n}%
     \selectfont #1}}
\newcommand{\Sha}{{\mbox{\textcyr{Sh}}}}

%\font\cyr=wncyr10 scaled \magstep 1
%\font\cyr=wncyr10

%\newcommand{\Sha}{{\cyr X}}
\newcommand{\Shaan}{\Sha_{\mbox{\tiny \rm an}}}
\newcommand{\TS}{Shafarevich-Tate group}

\newcommand{\Gam}{\Gamma}
\newcommand{\X}{\mathcal{X}}
\newcommand{\cH}{\mathcal{H}}
\newcommand{\cA}{\mathcal{A}}
\newcommand{\cF}{\mathcal{F}}
\newcommand{\cG}{\mathcal{G}}
\newcommand{\cJ}{\mathcal{J}}
\newcommand{\cL}{\mathcal{L}}
\newcommand{\cV}{\mathcal{V}}
\newcommand{\cO}{\mathcal{O}}
\newcommand{\cQ}{\mathcal{Q}}
\newcommand{\cX}{\mathcal{X}}
\newcommand{\ds}{\displaystyle}
\newcommand{\M}{\mathcal{M}}
\newcommand{\E}{\mathcal{E}}
\renewcommand{\L}{\mathcal{L}}
\newcommand{\J}{\mathcal{J}}
\DeclareMathOperator{\new}{new}
\DeclareMathOperator{\Morph}{Morph}
\DeclareMathOperator{\old}{old}
\DeclareMathOperator{\Sym}{Sym}
\DeclareMathOperator{\Symb}{Symb}
%\newcommand{\Sym}{\mathcal{S}{\rm ym}}
\newcommand{\dw}{\delta(w)} 
\newcommand{\dwh}{\widehat{\delta(w)}}      
\newcommand{\dlwh}{\widehat{\delta_\l(w)}} 
\newcommand{\dash}{-\!\!\!\!-\!\!\!\!-\!\!\!\!-} 
\DeclareMathOperator{\tor}{tor}  
\newcommand{\Frobl}{\Frob_{\ell}}
\newcommand{\tE}{\tilde{E}}
\renewcommand{\l}{\ell}
\renewcommand{\t}{\tau}
\DeclareMathOperator{\cond}{cond}
\DeclareMathOperator{\Spec}{Spec}
\DeclareMathOperator{\Div}{Div}
\DeclareMathOperator{\Jac}{Jac}
\DeclareMathOperator{\res}{res}
\DeclareMathOperator{\Ker}{Ker}
\DeclareMathOperator{\Coker}{Coker}
\DeclareMathOperator{\sep}{sep}
\DeclareMathOperator{\sign}{sign}
\DeclareMathOperator{\unr}{unr}
\newcommand{\N}{\mathcal{N}}
\newcommand{\U}{\mathcal{U}}
\newcommand{\Kbar}{\overline{K}}
\newcommand{\Lbar}{\overline{L}}
\newcommand{\gammabar}{\overline{\gamma}}
\newcommand{\q}{\mathbf{q}}
%\renewcommand{\star}{\times}
\newcommand{\gM}{\mathfrak{M}}
\newcommand{\gA}{\mathfrak{A}}
\newcommand{\gP}{\mathfrak{P}}
\newcommand{\bmu}{\boldsymbol{\mu}}
\newcommand{\union}{\cup}
\newcommand{\Tl}{T_{\ell}}
\newcommand{\into}{\rightarrow}
\newcommand{\onto}{\twoheadrightarrow}%  Surjection arrow

\newcommand{\meet}{\cap}
\newcommand{\cross}{\times}
\DeclareMathOperator{\md}{mod}
\DeclareMathOperator{\toric}{toric}
\DeclareMathOperator{\tors}{tors}
\DeclareMathOperator{\Frac}{Frac}
\DeclareMathOperator{\corank}{corank}
\newcommand{\rb}{\overline{\rho}}
\newcommand{\ra}{\rightarrow}
\newcommand{\xra}[1]{\xrightarrow{#1}}
\newcommand{\hra}{\hookrightarrow}
\newcommand{\la}{\leftarrow}
\newcommand{\lra}{\longrightarrow}
\newcommand{\riso}{\xrightarrow{\sim}}
\newcommand{\da}{\downarrow}
\newcommand{\ua}{\uparrow}
\newcommand{\con}{\equiv}
\newcommand{\Gm}{\mathbb{G}_m}
\newcommand{\pni}{\par\noindent}
\newcommand{\set}[1]{\{#1\}}
\newcommand{\iv}{^{-1}}
\newcommand{\alp}{\alpha}
\newcommand{\bq}{\mathbf{q}}
\newcommand{\cpp}{{\tt C++}}
\newcommand{\tensor}{\otimes}
\newcommand{\bg}{{\tt BruceGenus}}
\newcommand{\abcd}[4]{\left(
        \begin{smallmatrix}#1&#2\\#3&#4\end{smallmatrix}\right)}
\newcommand{\mthree}[9]{\left(
        \begin{matrix}#1&#2&#3\\#4&#5&#6\\#7&#8&#9
        \end{matrix}\right)}
\newcommand{\mtwo}[4]{\left(
        \begin{matrix}#1&#2\\#3&#4
        \end{matrix}\right)}
\newcommand{\vtwo}[2]{\left(
        \begin{matrix}#1\\#2
        \end{matrix}\right)}
\newcommand{\smallmtwo}[4]{\left(
        \begin{smallmatrix}#1&#2\\#3&#4
        \end{smallmatrix}\right)}
\newcommand{\twopii}{2\pi{}i{}}  
\newcommand{\eps}{\varepsilon}
\newcommand{\vphi}{\varphi}
\newcommand{\gp}{\mathfrak{p}}
\newcommand{\W}{\mathcal{W}}
\newcommand{\oz}{\overline{z}}
\newcommand{\Zpstar}{\Zp^{\star}}
\newcommand{\Zhat}{\widehat{\Z}}
\newcommand{\Zbar}{\overline{\Z}}
\newcommand{\Zl}{\Z_{\ell}}
\newcommand{\comment}[1]{}
\newcommand{\Q}{\mathbb{Q}}
\newcommand{\QQ}{\mathbb{Q}}
\newcommand{\GQ}{G_{\Q}}
\newcommand{\R}{\mathbb{R}}
\newcommand{\RR}{\mathbb{R}}
\newcommand{\PP}{\mathbb{P}}
\newcommand{\D}{{\mathbf D}}
\newcommand{\cC}{\mathcal{C}}
\newcommand{\cD}{\mathcal{D}}
\newcommand{\cP}{\mathcal{P}}
\newcommand{\cS}{\mathcal{S}}
\newcommand{\Sbar}{\overline{S}}
\newcommand{\K}{{\mathbb K}}
\newcommand{\C}{\mathbb{C}}
\newcommand{\CC}{\mathbb{C}}
\newcommand{\Cp}{{\mathbb C}_p}
\newcommand{\Sets}{\mbox{\rm\bf Sets}}
\newcommand{\bcC}{\boldsymbol{\mathcal{C}}}
\renewcommand{\P}{\mathbb{P}}
\newcommand{\Qbar}{\overline{\Q}}
\newcommand{\QQbar}{\overline{\Q}}
\newcommand{\kbar}{\overline{k}}
\newcommand{\dual}{\bot}
\newcommand{\T}{\mathbb{T}}
\newcommand{\TT}{\mathbb{T}}
\newcommand{\calT}{\mathcal{T}}
\newcommand{\cT}{\mathcal{T}}
\newcommand{\cbT}{\mathbb{\mathcal{T}}}
\newcommand{\cU}{\mathcal{U}}
\newcommand{\Z}{\mathbb{Z}}
\newcommand{\ZZ}{\mathbb{Z}}
\newcommand{\F}{\mathbb{F}}
\newcommand{\FF}{\mathbb{F}}
\newcommand{\Fl}{\F_{\ell}}
\newcommand{\Fell}{\Fl}
\newcommand{\Flbar}{\overline{\F}_{\ell}}
\newcommand{\Flnu}{\F_{\ell^{\nu}}}
\newcommand{\Fbar}{\overline{\F}}
\newcommand{\Fpbar}{\overline{\F}_p}
\newcommand{\fbar}{\overline{f}}
\newcommand{\Qp}{\Q_p}
\newcommand{\Ql}{\Q_{\ell}}
\newcommand{\Qell}{\Q_{\ell}}
\newcommand{\Qlbar}{\overline{\Q}_{\ell}}
\newcommand{\Qlnr}{\Q_{\ell}^{\text{nr}}}
\newcommand{\Qlur}{\Q_{\ell}^{\text{ur}}}
\newcommand{\Qltm}{\Q_{\ell}^{\text{tame}}}
\newcommand{\Qv}{\Q_v}
\newcommand{\Qpbar}{\Qbar_p}
\newcommand{\Zp}{\Z_p}
\newcommand{\Fp}{\F_p}
\newcommand{\Fq}{\F_q}
\newcommand{\Fqbar}{\overline{\F}_q}
\newcommand{\Ad}{Ad}
\newcommand{\adz}{\Ad^0}
\renewcommand{\O}{\mathcal{O}}
\newcommand{\A}{\mathcal{A}}
\newcommand{\Og}{O_{\gamma}}
\newcommand{\isom}{\cong}
\newcommand{\ncisom}{\approx}   % noncanonical isomorphism
\DeclareMathOperator{\ab}{ab}
\DeclareMathOperator{\alg}{alg}
\DeclareMathOperator{\Aut}{Aut}
\DeclareMathOperator{\Frob}{Frob}
\DeclareMathOperator{\Fr}{Fr}
\DeclareMathOperator{\Ver}{Ver}
\DeclareMathOperator{\Norm}{Norm}
\DeclareMathOperator{\Ind}{Ind}
\DeclareMathOperator{\norm}{norm}
\DeclareMathOperator{\disc}{disc}
\DeclareMathOperator{\ord}{ord}
\DeclareMathOperator{\GL}{GL}
\DeclareMathOperator{\PSL}{PSL}
\DeclareMathOperator{\PGL}{PGL}
\DeclareMathOperator{\Gal}{Gal}
\DeclareMathOperator{\SL}{SL}
\DeclareMathOperator{\SO}{SO}
\DeclareMathOperator{\WC}{WC}
\newcommand{\galq}{\Gal(\Qbar/\Q)}
\newcommand{\rhobar}{\overline{\rho}}
\newcommand{\cM}{\mathcal{M}}
\newcommand{\cB}{\mathcal{B}}
\newcommand{\cE}{\mathcal{E}}
\newcommand{\cR}{\mathcal{R}}
\newcommand{\et}{\text{\rm\'et}}

\newcommand{\sltwoz}{\SL_2(\Z)}
\newcommand{\sltwo}{\SL_2}
\newcommand{\gltwoz}{\GL_2(\Z)}
\newcommand{\mtwoz}{M_2(\Z)}
\newcommand{\gltwoq}{\GL_2(\Q)}
\newcommand{\gltwo}{\GL_2}
\newcommand{\gln}{\GL_n}
\newcommand{\psltwoz}{\PSL_2(\Z)}
\newcommand{\psltwo}{\PSL_2}
\newcommand{\h}{\mathfrak{h}}
\renewcommand{\a}{\mathfrak{a}}
\newcommand{\p}{\mathfrak{p}}
\newcommand{\m}{\mathfrak{m}}
\newcommand{\trho}{\tilde{\rho}}
\newcommand{\rhol}{\rho_{\ell}}
\newcommand{\rhoss}{\rho^{\text{ss}}}
\DeclareMathOperator{\tr}{tr}
\DeclareMathOperator{\order}{order}
\DeclareMathOperator{\ur}{ur}
\DeclareMathOperator{\Tr}{Tr}
\DeclareMathOperator{\Hom}{Hom}
\DeclareMathOperator{\Mor}{Mor}
\DeclareMathOperator{\HH}{H}
\renewcommand{\H}{\HH}
\DeclareMathOperator{\Ext}{Ext}
\DeclareMathOperator{\Tor}{Tor}
\newcommand{\smallzero}{\left(\begin{smallmatrix}0&0\\0&0
                        \end{smallmatrix}\right)}
\newcommand{\smallone}{\left(\begin{smallmatrix}1&0\\0&1
                        \end{smallmatrix}\right)}

\newcommand{\pari}{{\sc Pari}}
\newcommand{\magma}{{\sc Magma}}
\newcommand{\hecke}{{\sc Hecke}}
\newcommand{\lidia}{{\sc LiDIA}}

%%%% Theoremstyles
\theoremstyle{plain}
\newtheorem{theorem}{Theorem}[section]
\newtheorem{proposition}[theorem]{Proposition}
\newtheorem{corollary}[theorem]{Corollary}
\newtheorem{claim}[theorem]{Claim}
\newtheorem{lemma}[theorem]{Lemma}
\newtheorem{hypothesis}[theorem]{Hypothesis}
\newtheorem{conjecture}[theorem]{Conjecture}

\theoremstyle{definition}
\newtheorem{definition}[theorem]{Definition}
\newtheorem{question}[theorem]{Question}
\newtheorem{idea}[theorem]{Idea}
\newtheorem{project}[theorem]{Project}
\newtheorem{problem}[theorem]{Problem}
\newtheorem{openproblem}[theorem]{Open Problem}
\newtheorem{challenge}[theorem]{Challenge}

%\theoremstyle{remark}
\newtheorem{goal}[theorem]{Goal}
\newtheorem{remark}[theorem]{Remark}
\newtheorem{remarks}[theorem]{Remarks}
\newtheorem{example}[theorem]{Example}
\newtheorem{exercise}[theorem]{Exercise}

\numberwithin{equation}{section}
\numberwithin{figure}{section}
\numberwithin{table}{section}


% bulleted list environment
\newenvironment{bulletlist}
   {
      \begin{list}
         {$\bullet$}
         {
            \setlength{\itemsep}{.5ex}
            \setlength{\parsep}{0ex}
            \setlength{\parskip}{0ex}
            \setlength{\topsep}{.5ex}
         }
   }
   {
      \end{list}
   }
%end newenvironment

% bulleted list environment
\newenvironment{dashlist}
   {
      \begin{list}
         {---}
         {
            \setlength{\itemsep}{.5ex}
            \setlength{\parsep}{0ex}
            \setlength{\parskip}{0ex}
            \setlength{\topsep}{.5ex}
         }
   }
   {
      \end{list}
   }
%end newenvironment

% numbered list environment
\newcounter{listnum}
\newenvironment{numlist}
   {
      \begin{list}
            {{\em \thelistnum.}}{
            \usecounter{listnum}
            \setlength{\itemsep}{.5ex}
            \setlength{\parsep}{0ex}
            \setlength{\parskip}{0ex}
            \setlength{\topsep}{.5ex}
         }
   }
   {
      \end{list}
   }
%end newenvironment

\newcommand{\hd}[1]{\vspace{1ex}\noindent{\bf #1} }
\newcommand{\nf}[1]{\underline{#1}} 
\newcommand{\cbar}{\overline{c}}

\DeclareMathOperator{\rad}{rad}

\theoremstyle{definition}
\newtheorem{algor}[theorem]{Algorithm}
\newenvironment{algorithm}[1]{%
\begin{algor}[#1]\index{{\bf Algorithm}!#1}
}%
{\end{algor}}

\newenvironment{steps}%
{\begin{enumerate}\setlength{\itemsep}{0.1ex}}{\end{enumerate}}

\usepackage{color}
\usepackage{listings} 
\lstdefinelanguage{Sage}[]{Python}
{morekeywords={True,False,sage,singular},
sensitive=true}
\lstset{
  showtabs=False,
  showspaces=False,
  showstringspaces=False,
  commentstyle={\ttfamily\color{dredcolor}},
  keywordstyle={\ttfamily\color{dbluecolor}\bfseries},
  stringstyle ={\ttfamily\color{dgraycolor}\bfseries},
  language = Sage,
  basicstyle={\small \ttfamily},
  aboveskip=1em,
  belowskip=1em,
  backgroundcolor=\color{lightyellow},
  frame=single
}
\definecolor{lightyellow}{rgb}{1,1,.86}
\definecolor{dblackcolor}{rgb}{0.0,0.0,0.0}
\definecolor{dbluecolor}{rgb}{.01,.02,0.7}
\definecolor{dredcolor}{rgb}{0.8,0,0}
\definecolor{dgraycolor}{rgb}{0.30,0.3,0.30}
\definecolor{graycolor}{rgb}{0.35,0.35,0.35}
\newcommand{\dblue}{\color{dbluecolor}\bf}
\newcommand{\dred}{\color{dredcolor}\bf}
\newcommand{\dblack}{\color{dblackcolor}\bf}
\newcommand{\gray}{\color{graycolor}}

\newcommand{\dbd}[1]{\langle#1\rangle}   % make a diamond bracket d symbol

%%% Local Variables: 
%%% mode: latex
%%% TeX-master: t
%%% End: 






\usepackage{hyperref}
\usepackage{multirow}
\newcommand{\n}{\mathfrak{n}}
\newcommand{\cN}{\mathcal{N}}
\newcommand{\ap}[1]{a_{\p_{#1}}}
\newcommand{\Ebar}{\overline{E}}
\newcommand{\round}[1]{\left\lfloor{#1}\right\rceil}
\newcommand{\fc}{\mathfrak{c}}
\newcommand{\OF}{\cO_F}
\newcommand{\dembele}{Demb\'el{\'e}\xspace}
\input cyracc.def %sha
\font\tencyr=wncyr10
%\def\Sha{\text{\tencyr\cyracc{Sh}}}
\def\smallsha{\text{\eightcyr\cyracc{Sh}}}
%
%Notatation:
%   F = \Q(\sqrt 5)
%   \cO_F = \OF = ring of integers in F
%   \varphi = (1 + \sqrt 5)/2
%   \sigma_1(\varphi) = 1.618...
%   \sigma_2(\varphi) = -.618...
%   \p = prime ideal in \cO_F
%   \n is also an ideal
%   \m is also an ideal
%   \fc (frak c) is also an ideal
%   \ap[n] = Fourier coefficient of f at \p_{n}
%   galois conjugate of E = \Ebar
%   \dembele = Dembele


\renewcommand{\phi}{\varphi}
\title{A Database of Elliptic Curves over $\Q(\sqrt{5})$---First Report}

\author[Bober et al.]{Jon Bober, Alyson Deines, Ariah Klages-Mundt, Ben
  LeVeque, R. Andrew Ohana, Ashwath Rabindranath, Paul Sharaba, William
  Stein}
\thanks{ This work is supported by NSF grant DMS-0757627, administered
by the American Institute of Mathematics. }
\begin{document}

\begin{abstract} 
We describe a tabulation of (conjecturally) modular
elliptic curves over the field $\Q(\sqrt5)$ up to the first curve of rank $2$. Using an
efficient implementation of an algorithm of Lassina \dembele
\cite{dembele:hilbert5}, we computed large tables of Hilbert modular
forms of weight $(2,2)$ over $\Q(\sqrt 5)$, and via a variety of ad hoc
methods we constructed corresponding elliptic curves, including
(again, conjecturally) all elliptic curves over $\Q(\sqrt5)$ which have
conductor with norm less than 1831. 
\end{abstract} 

\maketitle

\section{Introduction}\label{sec:intro}

\subsection{Elliptic Curves over $\Q$}
Tables of elliptic curves over $\Q$ have been of great value in
mathematical research.  Some of the first such tables were those in
Antwerp IV \cite{antwerpiv}, which included all elliptic curves over
$\Q$ of conductor up to $200$, and also a table of all elliptic curves
with bad reduction only at $2$ and $3$.  

Cremona's book \cite{cremona:algs} gives a detailed description of
algorithms that together output a list of all elliptic curves over
$\Q$ of any given conductor, along with extensive data about each
curve.  The proof that his algorithm output {\em all} curves of given
conductor had to wait for the proof of the full modularity theorem in
\cite{breuil-conrad-diamond-taylor}.  Cremona has subsequently
computed tables \cite{cremona:onlinetables} of all elliptic curves
over $\Q$ of conductor up to $220,\!000$, including Mordell-Weil
groups and other extensive data about each curve; he expects to soon
reach his current target, conductor $234,\!446$, which is the smallest
known conductor of a rank $4$ curve.

In a different direction, Stein-Watkins (see \cite{stein-watkins:ants5, bmsw:bulletins}) 
created a table of 136,832,795 elliptic curves over $\Q$ of conductor $\leq 10^8$, and a
table of 11,378,911 elliptic curves over $\Q$ of prime conductor $\leq
10^{10}$. 
% For each curve, these tables contain a numerical
%approximation of $\ord_{s=1} L(E,s)$, and some other data, though the
%actual ranks of every one of these curves has not been determined.
There are many curves of large discriminant missing from the
Stein-Watkins tables, since these tables are made by enumerating
curves with relatively small defining equations, and discarding those
of large conductor, rather than systematically finding all curves of
given conductor no matter how large the defining equation.

 
\subsection{Why $\Q(\sqrt{5})$?}

Like $\Q$, the field $F=\Q(\sqrt{5})$ is a totally real field, and
many of the theorems and ideas about elliptic curves over $\Q$ have
been generalized to totally real fields. As is the case over $\Q$,
there is a notion of modularity coming from Shimura curve
parametrizations and Hilbert modular forms.
Moreover, work of Zhang \cite{zhang:heightsshimura}  has extended many
results of Gross-Zagier \cite{gross-zagier} and Kolyvagin
\cite{kolyvagin:mordellweil} to the context of elliptic curves over
totally real fields.

If we order totally real number fields $K$ by the absolute value of
their discriminant, then $F=\Q(\sqrt{5})$ comes next after $\Q$ (the
Minkowski bound implies that $|D_K| \geq (n^n/n!)^2$, where
$n=[K:\Q]$, so if $n\geq 3$ then $|D_K|>20$). Since $5$ divides
$\disc(F)=5$,  the method of Taylor-Wiles does not generalize easily to elliptic
 curves over $F$, thus making $\Q(\sqrt{5})$ even more interesting. 
The field $F$ also has $31$ CM $j$-invariants, which is far more than
 any other quadratic field (see Section~\ref{sec:cm}).  Letting
 $\vphi=\frac{1+\sqrt{5}}{2}$, we have that the group of units
 $\{\pm 1\} \times \langle \vphi \rangle$ of the
ring $R=\cO_F=\Z[\vphi]$ of integers of $F$ is infinite, leading to
additional complications.  Finally, $F$ has even degree, which
makes certain computations more difficult, since the cohomological
techniques of \cite{greenberg-voight:shimura} are not available. 


\subsection{Modularity conjecture}\label{sec:mod}
The following conjecture is open:
\begin{conjecture}[Modularity]\label{conj:mod}
  The set of $L$-functions of elliptic curves over $F$ equals the set
  of $L$-functions associated to cuspidal Hilbert modular newforms
  over $F$ of weight $(2,2)$ with rational Hecke eigenvalues.
\end{conjecture}
Given the huge amount of progress on modularity theorems initiated by
\cite{wiles:fermat}, we are optimistic that Conjecture~\ref{conj:mod}
will be proved.  {\em We thus officially assume
  Conjecture~\ref{conj:mod} for the rest of this paper.} \\

{\bf Acknowledgements.} We would like to thank John Cremona, Tom
Fisher, Noam Elkies, and Richard Taylor for helpful conversations.  We
would like to especially thank Joanna Gaski for providing (via the
method of Section~\ref{sec:naive}) the explicit table of elliptic
curves that kickstarted this project.

\section{Computing Hilbert modular forms over $F$}\label{sec:hmf}

In this section, we describe \dembele's approach to computing Hilbert modular forms over
$F$, but with additional remarks that fill in missing details in the
algorithm and address how to create a fast implementation.

\subsection{Hilbert modular forms and quaternion 
algebras}\label{sec:dembele}
\dembele \cite{dembele:hilbert5} introduced an algebraic approach via
the Jacquet-Langlands correspondence to computing Hilbert modular
forms of weight $(2,2)$ over $F$.  The Hamiltonian quaternion algebra
$F[i,j,k]$ over $F$ is ramified exactly at the two infinite places,
and contains the maximal order
$$
 S = R\left[\frac{1}{2}(1-\overline{\vphi} i + \vphi j),\,
       \frac{1}{2}(-\overline{\vphi} i + j + \vphi k),\,
       \frac{1}{2}(\vphi i - \overline{\vphi} j + k), \,
       \frac{1}{2}(i + \vphi j - \overline{\vphi} k)\right].
$$
For any nonzero ideal $\n$ in $R=\cO_F$,
let $\P^1(R/\n)$ by the set of equivalence classes of
 column vectors with two coprime entries $a,b \in R/\n$ modulo the
 action of $(R/\n)^*$.  We use the notation $[a:b]$
to denote the equivalence class of 
$\left(\begin{smallmatrix}a\\b\end{smallmatrix}\right)$.
The algebra $F[i,j,k]$ is unramified at all finite primes and ramified
at the two real infinite primes; in
particular, for each prime $\p\mid \n$, we may fix a choice of isomorphism
$F[i,j,k]\tensor F_{\p} \ncisom M_2(F_{\p})$, which induces a left
action of $S^*$ on $\P^1(R/\n)$. The Jaquet-Langlands correspondence implies that
the space of Hilbert modular forms of level $\n$ and weight $(2,2)$ is
noncanonically isomorphic as a module over the Hecke algebra
$$\T=\Z[T_\p :  \p \text{ nonzero prime ideal of }R]$$ 
to the finite dimensional complex vector space $ V = \C[S^* \backslash
\P^1(R/\n)].  $ The action of $T_{\p}$, for $p\nmid \n$, is
$T_{\p}([x]) = \sum [\alpha x]$, where the sum is over the classes
$[\alpha]\in S/S^*$ with $N_{\text{red}}(\alpha)=\pi_{\p}$ (reduced quaternion norm),
where $\pi_{\p}$ is a fixed choice of totally positive generator of~$\p$.

\subsection{Remarks on Computing with $\P^1(R/\n)$}

In order to implement the algorithm sketched in
Section~\ref{sec:dembele}, it is critical that we can compute with
$\P^1(R/\n)$ very, very quickly.  For example, to apply the method of
Section~\ref{sec:specialvalues} below, in some cases we have to
compute tens of thousands of Hecke operators.  Thus in this section we
make some additional remarks about how we did this fast
implementation.

When $\n=\p^e$ is a prime power, it is
straightforward to efficiently enumerate representative elements of
$\P^1(R/\p^e)$, since each element $[x:y]$ of $\P^1(R/\p^e)$ has a
unique representative of the form $[1:b]$ or $[a:1]$ with $a$
divisible by $\p$, and these are all distinct.  It is easy 
to put  any $[x:y]$ 
in this canonical form and enumerate the elements of $\P^1(R/\p^e)$, after
choosing a way to enumerate the elements of $R/\p^e$.
An enumeration of $R/\p^e$ is easy to give
once we decide on how to represent $R/\p^e$.

In general, factor $\n = \prod_{i=1}^m \p_i^{e_i}$. 
We have a bijection $\P^1(R/\n) \isom \prod_{i=1}^m
\P^1(R/\p_i^{e_i})$, which allows us to reduce to the prime power
case, at the expense of having to compute the bijection 
$R/\n \isom \prod R/\p_i^{e_i}$.
To this end, we {\em represent elements} of $R/\n$ as
$m$-tuples in $\prod R/\p_i^{e_i}$, thus making
computation of the bijection trivial.  

To minimize dynamic memory allocation, thus speeding up the
code by an order of magnitude, in the implementation
we make some arbitrary
bounds; this is not a serious constraint, since the linear algebra
needed to isolate eigenforms for levels beyond this bound is
prohibitive. We assume $m\leq 16$ and each individual 
$p_i^{e_i}\leq 2^{31}$, where $p_i$ is the residue characteristic of $\p_i$.  
In all cases, we represent an
element of $R/\p_i^{e_i}$ as a pair of  64-bit integers, and
represent an element of $R/\n$ as an array of 16 pairs of 64-bit
integers. We use this representation in all cases, even if $\n$ is
divisible by less than $16$ primes; the gain in speed coming
from avoiding dynamic memory allocation more than compensates for the
wasted memory.

Let $\p^e$ be one of the prime power factors of $\n$, and let $p$ be the residue
characteristic of $\p$. We have one of the following cases:
\begin{itemize}
\item $\p$ splits in $R$; then $R/\p\isom \Z/p\Z$ and we represent elements
of $R/\p^e$ as pairs $(a,0)$ mod $p^e$ with the usual addition and multiplication
in the first factor.
\item $\p$ is inert in $R$; then $R/\p^e\isom (\Z/p^e\Z)[x]/(x^2-x-1)$,
and we represent elements by pairs $(a,b) \in \Z/p^e\Z$ with multiplication
$$(a,b)(c,d) = (ac+bd,ad+bd+bc) \mod p^e.$$
\item $\p$ is ramified and $e=2f$ is even; this is exactly the
same as the case when $\p$ is inert but with $e$ replaced by $f$,
since $R/\p^eR \isom (\Z/p^{f}\Z)[x]/(x^2-x-1)$.
\item $\p$ is ramified (so $p=5$) and $e=2f-1$ is odd; the ring $A=R/\p^e$
  is trickier than the rest, because it is {\em not} of the form
  $\Z[x]/(m,g)$ where $m\in\Z$ and $g \in\Z[x]$.  We have $A \ncisom
  (\Z/5^f\Z)[x]/(x^2-5, 5^{f-1}x)$, and represent elements of $A$ as
  pairs $(a,b) \in (\Z/5^f)\times (\Z/5^{f-1}\Z)$, with arithmetic
  given by
\begin{align*}
(a,b) + (c,d) &= (a+c \mod 5^f,\,\,\, b+d \mod 5^{f-1})\\
(a,b)\cdot (c,d) &= (ac+5bd \mod 5^f,\,\,\, ad+bc \mod 5^{f-1}).
\end{align*}
 We find that $ \varphi \in R \mapsto (1/2,1/2)$.
\end{itemize}



\section{Strategies for finding an elliptic curve attached to a Hilbert modular form}\label{sec:finding}
In this section we describe various strategies to find an elliptic curve
 associated to each of the Hilbert modular forms computed in Section 2.
Let $f$ be a rational cuspidal Hilbert newform of weight $(2,2)$ as
in Section~\ref{sec:hmf}.  According to Conjecture~\ref{conj:mod},
there is some elliptic curve $E_f$ over $f$ such that $L(f,s) =
L(E_f,s)$.  Note that $E_f$ is only well defined up to isogeny.
Unlike the case for elliptic curves over $\Q$ (see \cite{cremona:algs}), 
there appears to be no {\em efficient} direct algorithm to find $E_f$.
 Nonetheless, there are several approaches coming from various directions, 
which are each efficient in some cases.

Everywhere below, we continue to assume that Conjecture~\ref{conj:mod}
is true and assume that we have computed (as in Section~\ref{sec:hmf})
the Hecke eigenvalues $a_\p\in \Z$ of all rational Hilbert newforms of
some level $\n$, for $\Norm(\p)\leq B$ a good prime, where $B$ is
large enough distinguish newforms. In some cases we will need far more
$a_{\p}$ in order to distinguish between elements of our finite set of
newforms and furthermore, to compute with a sufficient degree of
precision, the $L$-function attached to each element.  We will also
need the $a_{\p}$ for bad $\p$ in a few cases, which we obtain
using the functional equation for the $L$-function.

\subsection{Extremely naive enumeration}\label{sec:naive}

The most naive strategy is to systematically enumerate 
 elliptic curves $E: y^2 = x^3 + ax + b$, with $a,b\in R$,
and for each $E$, to compute $a_\p(E)$ for $\p$ not dividing $\Disc(E)$
 by counting points on $E$ reduced modulo $\p$.  If all the $a_\p(E)$ match
 with those of the input newform $f$ up to the bound $B$, we then compute
 the conductor $\cN_E$, and if it equals $\n$, we conclude from the sufficient 
largeness of $B$ that $E$ is in the isogeny class of $E_f$.

Under our hypotheses, this approach provides a deterministic and 
terminating algorithm to find all $E_f$. However, it can be extremely slow
 when $\n$ is small but the simplest curve in the isogeny class 
of $E_f$ has large coefficients.  For example, using this search method it would be infeasible to find the curve \eqref{equation:fisher-curve}
computed by Fisher using the visibility of \Sha$[7]$.

\subsection{Sieved enumeration}

A refinement to the approach discussed above uses the $a_\p$ values to impose
congruence conditions modulo $\p$ on $\#\tilde E(R/\p)$.
Letting $f$ be a newform computed in \ref{sec:hmf} with Hecke eigenvalues
$a_\p$, we know that $\#\tilde E_f(R/\p)=\bN(\p)+1-a_{\p}$.
Given $\p$ not dividing the level $\n$, we can find all elliptic curves
modulo $\p$ with the specified number of points, especially when $\bN(\p) +1 - a_\p$ has
few prime factors. We impose these congruence conditions at multiple primes $\p_i$,
use the Chinese Remainder Theorem, and lift the resulting curves modulo $R/ (\prod \p_i)$
to non-singular curves over $R$.

While this method, like the previous one, will eventually terminate, it too is very
ineffective if every $E$ in the class of isogenous curves corresponding to $f$ has
large coefficients. However in practice, by optimally choosing the number of primes $\p_i$, 
a reasonably efficient implementation of this method can be obtained.
% Thus this is
%a useful technique when we are building tables of elliptic curves
%associated to newforms and would like to reduce the number of newforms
%that we must run through slower techniques.

\subsection{Torsion families}\label{sec:torsion}

We find elliptic curves of small conductor by specializing explicit parametrizations
of families of elliptic curves over $F$ having specified torsion subgroups.
We use the parametrizations of \cite{kubert:torfam}.

\begin{theorem}[Kamienny-Najman, \cite{kamienny-najman}] The following is
a complete list of torsion structures for elliptic curves over $F$:

$$\begin{array}{lll}
\mathbb{Z}/m\mathbb{Z},   &1 \leq m \leq 10,& m = 12,\\
\mathbb{Z}/2\mathbb{Z} \oplus \mathbb{Z}/2m\mathbb{Z}, &  1 \leq m \leq 4,&\\
\mathbb{Z}/15\mathbb{Z}.&&
\end{array}$$ \end{theorem}

We use the following proposition to determine in which family to
search.
\begin{proposition}\label{prop:ptor}
Let $\ell$ be a prime and $E$ a curve over $F$.
Then $\ell \mid \# E'(F)_{\tor}$ for some curve $E'$ in the isogeny class
of $E$ if and only if $\ell \mid \bN(\p)+1 - a_{\p}$ for all
odd primes $\p$ at which $E$ has good reduction.
\end{proposition}
\begin{proof}
If $\ell  \mid \# E'(F)_{\tor}$, from the injectivity of the
reduction map at good primes \cite{katz:torsion}[Appendix], we have that
$\ell \mid \#\tilde E'(\F_\p) = \bN(\p)+1 - a_{\p}$. The converse statement
is one of the main results of \cite{katz:torsion}.
\end{proof}

By applying Proposition~\ref{prop:ptor} for all $a_{\p}$ with $\p$ up
to some bound, we can decide whether or not it is {\em likely} that
some curve in the isogeny class of $E$ contains an $F$-rational
$\ell$-torsion point. If this is the case, then we search over those
families of curves with rational $\ell$-torsion. With a relatively small
search space, we thus find many curves with large coefficients more
quickly than with the algorithm of Section~\ref{sec:naive}.
For example, we first found the curve $E$ given by 
$$y^2 + \vphi{}y = x^3 + \left(27 \vphi{} - 43\right)x + \left(-80 \vphi{} + 128\right) 
$$ with norm conductor $145$ by searching for curves with torsion subgroup $\Z/7\Z$.


\subsection{Twisting}

Let $E$ be an elliptic curve $F$. A \emph{twist} $E'$ of $E$ is a
curve over $F$ that is isomorphic to $E$ over some extension of $F$. A
\emph{quadratic twist} is a twist in which the extension has degree
$2$.  We can use twisting to find curves that may otherwise be
difficult to find as follows: starting with a known elliptic curve $E$
of some (small) conductor, we compute its twists of conductor up to
some prespecified bound, and add them to our table.

More explicitly, if $E$ is given by $y^2=x^3+ax+b$ and $d\in F^*$,
then the twist $E^d$ of $E$ by $d$ is given by $dy^2=x^3+ax+b$; in
particular, we may assume that $d$ is square free.  The following
is well known:
\begin{proposition}\label{twist:bound}
If $\n$ is the
conductor of $E$ and $d \in F^*$ is coprime to $\n$, then the
conductor of $E^d$ is divisible by $d^2\n$.
\end{proposition}
\begin{proof}
There are choices of Weierstrass equations such that 
$\Delta(E^d) = 2^{12} d^6 \Delta(E)$, where $\Delta$
is the discriminant.
%var('a1,a2,a3,a4,a6,d')
%E = EllipticCurve([a1,a2,a3,a4,a6])
%factor(E.quadratic_twist(d).discriminant() / E.discriminant())
Thus the curve $E^d$ has bad reduction at each prime that divides $d$,
because twisting introduces a $6$th power of the squarefree $d$ into
the discriminant, and $d$ is coprime to $\Delta(E)$, so no change of
Weierstrass equation can remove this $6$th power.  Moreover, $E^d$ is
isomorphic to $E$ over an extension of the base field, so $E^d$ has
potentially good reduction at each prime dividing $d$.  Thus the
reduction at each prime dividing $d$ is additive.  The conductor is
unchanged at the primes dividing $\n$ because of the formula relating
the conductor, discriminant and reduction type (see \cite[App.~C,\S15]{silverman:aec}),
that formation of N\'eron models commutes with unramified base change,
and the fact that at the primes that divide $\n$ the minimal discriminant of $E^d$ is
the same as that of $E$.
%OR
%
% is the discriminant of the cubic in the Weierstrass equation.
% The curve $E^d$ thus has bad reduction at each prime that divides $d$ since no change of
%  equation can remove the $d^6$-factor in $\Delta(E^d)$, $d$ being coprime to $\Delta(E)$,  
% Moreover since $E^d$ is isomorphic to $E$ over $F(\sqrt{d})$, it has
% potentially good thus additive reduction at primes dividing $d$. We know that
% $\text{ord}_\p(\n') = 1 + \text{ord}_\p(\Delta(E^d)) - m$, where $m$ is the number of
%  connected components of the special fibre of the N\'eron model of $E$ \cite{silverman:aec}[C.~\S15]. 
% Furthermore, the formation of the N\'eron model commutes with unramified base change and at $\nu|\n$, 
%  $\text{ord}_\nu(\Delta(E^d)) = \text{ord}_\nu(\Delta(E))$ for minimal $\Delta$. 
% It follows that for $\p|d$, $\text{ord}_\p(\n') \geq 2$ as desired. 
\end{proof}

To find all twists $E^d$ with norm conductor at most $B$, we twist $E$
by all $d$ of the form $\pm \vphi^{\delta} d_0 d_1$, where $\delta\in
\{0,1\}$, $d_0$ is a product of a fixed choice of generators for
the prime divisors of $\n$, 
$d_1$ is a squarefree product of a fixed choice of generators of 
primes not dividing $\n$, and
$|\bN(d_1)| \leq \sqrt{B/C}$, where $C$ is the norm of the product of
the primes that exactly divide $\n$. We know from \ref{twist:bound} that 
this search is exhaustive.


For example, let $E$ be given by $y^2 + xy + \vphi{}y = x^3 +
\left(-\vphi{} - 1\right)x^2 $ of conductor $5\vphi - 3$ having norm $31$.
Following the above strategy to find twists of norm conductor $\leq
B := 1831$, we have $C=31$ and square-free $d_1$ such that
$|\bN(d_1)| \leq \sqrt{B/C} \approx 7.6\ldots$. Thus $d_1 \in \{1, 2,
\vphi, 2\vphi\}$ and checking all possibilities for
$\vphi^{\delta} d_0 d_1$, we find the curve $E^{-\vphi-2}$ having
norm conductor $775$ and the curve $E^{5\vphi-3}$ having norm conductor
$961$. Other twists have larger norm conductors, e.g., $E^2$ has norm
conductor $126976=2^{12}\cdot 31$.


\subsection{Curves with good reduction outside $S$}

We use the algorithm of Cremona and Lingham from
\cite{cremona-lingham} to find all elliptic curves $E$ having good
reduction at primes outside of a finite set $\mathcal{S}$ of primes in
$F$. This algorithm has limitations over a general number
fields due to the difficulty of finding a generating set for $E(K)$ and
points on $E$ defined over $\cO_K$.
Using Cremona's $\textsc{Magma}$ implementation of the algorithm, we 
found several curves not found by other methods, e.g.,  
$y^2 + \left(\vphi{} +  1\right)xy + y = x^3 -x^2 + \left(-19 \vphi{} - 
 39\right)x + \left(-143 \vphi{} - 4\right)$ 
having norm conductor $1331$.

\subsection{Congruence families}
Suppose that we are searching for a curve $E$ and we already know
another curve $E'$ with $E[\ell] \ncisom E'[\ell]$, where $\ell$ is
some prime and $E[\ell]$ irreducible.  If $\ell=7,11$ then we can use
techniques of Fisher \cite{fisher:families_cong} to attempt to search
through the finitely many curves with mod~$\ell$ Galois representation
isomorphic to $E[\ell]$.
We used this approach to find the curve $E$ given by
\begin{equation}\label{equation:fisher-curve}
y^2 + \vphi{}xy = x^3 + \left(\vphi{} - 1\right)x^2 + \left(-257364 \vphi{} - 159063\right)x + \left(-75257037 \vphi{} - 46511406\right)
\end{equation}
with conductor $-38 \vphi + 10$, which has norm $1476$.  Given just the $a_{\p}$, we noticed that
$E[7] \ncisom E'[7]$, where $E'$ has norm conductor 369, then Fisher
used a \textsc{Magma} program to find rational points on a certain
quartic surface that parametrize curves with the same $E'[7]$.
Fortunately, our curve $E$ was amongst those curves.
We had already found $E'$ via a naive search, since it is given
by the equation $y^2 + \left(\vphi{} + 1\right)y = x^3 + \left(\vphi{} - 1\right)x^2 + \left(-2 \vphi{}\right)x$.


%Found using Cremona-Lingham:
%   1331 (33*a-11) (\varphi+1,-1,1,-19*\varphi-39,-143\varphi-4)
% The algorithm computes the finite $m$-Selmer groups $F(\mathcal{S},m)$ of
% $F^*$, where
% \begin{equation*}
%   F(\mathcal{S},m) = \{x\in F^*/(F^*)^m \ | \ ord_\mathfrak{p}(x) 
%     \equiv 0 \ (mod \ m) \ \ \forall \mathfrak{p}\notin \mathcal{S}\}\text{,}
% \end{equation*}
% for $m = 2, 3, 4, 6, \text{and}\ 12$. From these it computes a finite set of
% possible $j$-invariants such that each elliptic curve with good reduction
% outside $\mathcal{S}$ has $j$-invariant in this set. These $j$-values are either
% $j=0,1728$, cases which can be treated directly, or $\mathcal{S}$-integers in
% $F$ satisfying $w \equiv j^2(j-1728)^3 \ (\text{mod} \ F^{*6})$ for
% $w\in F(\mathcal{S},6)_{12}$, where $F(\mathcal{S},m)_{mn}$ is the image of the
% natural map $F(\mathcal{S},mn) \rightarrow F(\mathcal{S},m)$. In the latter
% case $j$ is of the form $j = \frac{x^3}{w} = 1728 + \frac{y^2}{w}$, where
% $(x,y)$ is an $\mathcal{S}$-integral point on the elliptic curve
% $E_w: Y^2 = X^3 - 1728w$. From this set of $j$-invariants, we construct each
% curve with the desired reduction properties (indeed, there are finitely many
% by Shafarevich's Theorem): for each $j = \frac{x^3}{w}$ (excluding
% $j = 0, 1728$, which are treated separately), we choose $u_0 \in F^*$ s.t.
% $(3u_0)^6w \in F(\mathcal{S},12)$, and each curve is of the form
% $E: Y^2 = X^3 - 3xu_0^2X - 2yu_0^3$. We must also check that each curve
% found has good reduction at the primes above 2 and 3 (if these primes
% are not in $\mathcal{S}$).

% Finding all $\mathcal{S}$-integral points over $F$, which give the
% $j$-invariants, presents a problem. For example, one technique for finding
% integral points is to find the generators of the curve's Mordell-Weil group.
% This can be difficult if these generators have large coefficients (this makes
% them hard to find with reasonable search bounds) or when the curve's rank
% is high (in this case, there are multiple generators to find). Due to these
% limitations, Cremona gives a naive search to directly find
% $\mathcal{S}$-integral points within a given search region, and so the number
% of curves that the program finds (and the speed with which it finds them)
% depends on the search bounds. Searches such as these are generally less
% effective over general number fields than over $\mathbb{Q}$. Indeed, the
% program failed to find a number of curves that we knew to exist by looking
% at corresponding modular forms and that we eventually found using other
% methods. However, the algorithm was successful in that it found some curves
% with very large $a$-invariants, which would be virtually impossible to find
% using a naive search.

\subsection{Special values of twisted $L$-series}\label{sec:specialvalues}
\newcommand{\Omegap}{\Omega^+}
\newcommand{\Omegam}{\Omega^-}
\newcommand{\Omegapp}{\Omega^{++}}
\newcommand{\Omegapm}{\Omega^{+-}}
\newcommand{\Omegamp}{\Omega^{-+}}
\newcommand{\Omegamm}{\Omega^{--}}
\newcommand{\OmegammEguess}{\Omega^{--}_{E, \mathrm{guess}}}
\newcommand{\OmegampEguess}{\Omega^{-+}_{E, \mathrm{guess}}}
\newcommand{\OmegapmEguess}{\Omega^{+-}_{E, \mathrm{guess}}}
\newcommand{\OmegappEguess}{\Omega^{++}_{E, \mathrm{guess}}}

In \cite{dembele:elliptic-curves-quadratic-fields}, Lassina \dembele
outlines some methods for finding modular elliptic curves from Hilbert
modular forms over real quadratic fields. Formally, these methods are
not proven to be any better than a direct search procedure, as they
involve making a large number of guesses, and a priori we do not know
just how many guesses we will need to make.  The method requires many
Fourier coefficients of the modular form $f$ that we have started
with, and computing these takes a lot of time. However, this method
certainly works extremely well in many cases, and after tuning it by
using large tables of curves that we had already computed, we we able
to use it to find more curves that we would have had no hope of
finding otherwise; we will give an example of one of these curves
later.  This section is longer than the others, because we created the
first general implementation of the algorithm from
\cite{dembele:elliptic-curves-quadratic-fields}.

When the level $\n$ is not square, \dembele's method relies on computing
or guessing periods of the curve by using special values of
$L$-functions of twists of the curve. In particular, the only inputs
required are the level of the Hilbert modular form and its L-series. So
we suppose that we know the level $\n = (N)$ of the form, where $N$ is
totally positive, and that we have sufficiently many coefficients of its
$L$-series $\ap{1}, \ap{2}, \ap{3}, \ldots$.

Let $\sigma_1$ and $\sigma_2$ denote the embeddings of $F$ into the
real numbers, with $\sigma_1(\varphi) \approx 1.61803398874989\ldots$. For
an elliptic curve $E$ over $F$ we get two associated embeddings into
the complex number, and hence a pair of period lattices. Let $\Omegap_E$
and $\Omegam_E$, which we refer to as the {\em periods} of $E$,
be the least real and imaginary periods of the lattice
which come from the embedding $\sigma_1$. As the period lattices are interchanged
when $E$ is replaced with its conjugate curve, we let $\Omegap_{\Ebar}$
and $\Omegam_{\Ebar}$ denote the least real and imaginary periods
of the lattice under the embedding associated to $\sigma_2$.

For ease, we write
\begin{align*}
\Omegapp_E &= \Omegap_E\Omegap_{\Ebar}& \ \Omegapm_E &= \Omegap_E\Omegam_{\Ebar} \\
\Omegamp_E &= \Omegam_E\Omegap_{\Ebar}& \ \Omegamm_E &= \Omegam_E\Omegam_{\Ebar}.
\end{align*}
We refer to these numbers as the {\em mixed periods} of $E$.

\subsubsection{Recovering the curve from its mixed periods}
If we know these mixed periods to sufficient precision, it is not hard
to recover the curve $E$. Without the knowledge of the discriminant
of the curve, we do not know the lattice type of the curve and its
conjugate, but there are only a few possibilities for what they might
be. This gives us a few possibilities for the $j$-invariant of
$E$. Observe that $\sigma_1(j(E))$ is either $j(\tau_1(E))$ or $j(\tau_2(E))$
and $\sigma_2(j(E))$ is either $j(\tau_1(\Ebar))$ or $j(\tau_2(\Ebar))$,
where
\begin{align*}
    \tau_1(E) &= \frac{\Omegamp_E}{\Omegapp_E} = \frac{\Omegam_E}{\Omegap_E}&  \tau_2(E) &= \frac{1}{2}\left(1 + \frac{\Omegamp_E}{\Omegapp_E}\right) = \frac{1}{2}\left(1 + \frac{\Omegam_E}{\Omegap_E}\right) \\
    \tau_1(\Ebar) &= \frac{\Omegapm_E}{\Omegapp_E} = \frac{\Omegam_E}{\Omegap_E} &    \tau_2(\Ebar) &= \frac{1}{2}\left(1 + \frac{\Omegapm_E}{\Omegapp_E}\right) = \frac{1}{2}\left(1 + \frac{\Omegam_{\Ebar}}{\Omegap_{\Ebar}}\right)
\end{align*}
and $j(\tau)$ is the modular $j$-function
\[
    j(\tau) = e^{-2\pi i \tau} + 744 + 196884e^{2\pi i \tau} + 21493760e^{4\pi i \tau} + \cdots
\]
We try each pair of possible embeddings for $j(E)$ in turn, and recognize
possibilities for $j(E)$ as an algebraic number. We then construct curves $E'$
corresponding to each possibility for $j(E)$. By computing a few $\ap{}(E)$, we
should be able to determine whether we have chosen the correct $j$-invariant
in which case $E'$ will be a twist of $E$. We can then recognize which twist it is
in order to recover $E$.

In practice, of course, as we have limited precision, and as $j(E)$ will not be
an algebraic integer it may not be feasible to directly determine its exact value,
especially if its denominator is large.
%Still, the above approach may be useful. UNNECESSARY?

To get around the problem of limited precision, we suppose that we have some extra information; 
namely, the discriminant $\Delta_E$ of the curve we are looking for.  With $\Delta_E$ in
hand we can directly determine which $\tau$ to choose: if $\sigma_1(\Delta_E) > 0$ then
$\sigma_1(j(E)) = j(\tau_1(E))$, and if $\sigma_1(\Delta_E) < 0$ then $\sigma_1(j(E)) = j(\tau_2(E))$,
and similarly for $\sigma_2$. We now compute $\sigma_1(c_4(E)) = (j(\tau) \sigma_1(\Delta_E))^{1/3}$
and $\sigma_2(c_4(E)) = (j(\tau') \sigma_2(\Delta_E))^{1/3}$.

Using the approximations of the two embeddings of $c_4$, we can recognize $c_4$ approximately
as an algebraic integer. Specifically, we compute
\[
    \alpha = \frac{\sigma_1(c_4) + \sigma_2(c_4)}{2}\ \ \  \text{and} \ \ \ 
    \beta = \frac{\sigma_1(c_4) - \sigma_2(c_4)}{2\sqrt{5}}.
\]
Then $c_4 = \alpha + \beta\sqrt{5}$, and we can find $c_6$.

\newcommand{\Deltaguess}{\Delta_{\textrm{guess}}}
\newcommand{\cfourguess}{c_{4,\textrm{guess}}}
\newcommand{\csixguess}{c_{6,\textrm{guess}}}
\newcommand{\Eguess}{E_\textrm{guess}}

In practice, there are two important difficulties we must overcome: we do not know
$\Delta_E$ and it may be quite difficult to get high precision approximations to the
mixed periods, and thus we may not be able to easily compute $c_4$. Thus, we actually
proceed by guessing a $\Deltaguess$ from which we
compute half-integers $\alpha$ and $\beta$ and an integer
$a + b\varphi \approx \alpha + \beta\sqrt5$, arbitrarily rounding either $a$ or $b$ if necessary.
We then make some choice of search range $M$, and for each pair of integers $m$ and $n$,
bounded in absolute value by $M$, we try each $\cfourguess = (a + m) + (b + n)\varphi$.

Given $\cfourguess$, we attempt to solve
\[
    \csixguess = \pm \sqrt{ \cfourguess^3 - 1728 \Deltaguess },
\]
and, if we can, we use these to construct a curve $\Eguess$. If $\Eguess$ has
the correct conductor and the correct Hecke eigenvalues, we declare that
we have found the correct curve; otherwise, we proceed to the next guess.

For a choice of $\Deltaguess$, we will generally start with the conductor $N_E$,
and then continue by trying unit multiples and by adding in powers of factors of $N_E$.

\subsubsection{Guessing the mixed periods}
We have thus far ignored the issue of actually finding the mixed periods of the curve that
we are looking for. Finding them presents an extra difficulty as our procedure involves
even more guesswork. The idea is to use special values of twists of the $L$-function
$L(f, s)$. Specifically, we twist by primitive quadratic Dirichlet characters over $\OF$,
which are homomorphisms $\chi : (\OF/\fc)^\times \rightarrow \pm1$, pulled back to $\OF$.

In the case of odd prime conductor, which we will stick to here, there is
just a single primitive quadratic character, namely the quadratic residue symbol.
A simple way to compute it is by making a
table of squares, or by choosing a primitive root of $g \in (\OF/\fc)^\times$, assigning
$\chi(g) = -1$, and again making a table by extending multiplicatively. Alternatively,
one could use a reciprocity formula as described in \cite{boylan-skoruppa:hecke-gauss-sums}.
For general conductor, one can compute with products of characters having prime conductor.

For a given $f$ and a primitive $\chi$, we can construct the twisted $L$-function
\[
    L(f, \chi, s) = \sum_{\m \subseteq \OF} \frac{\chi(m)a{_{\m}}}{N(\m)^s},
\]
where $m$ is a totally positive generator of $\m$. (Note that $\chi$ is not well defined
on ideals, but {\em is} well defined on totally positive generators of ideals.) $L(f, \chi, s)$
will satisfy a functional equation similar to that of $L(f, s)$, but the conductor is
multiplied by $\Norm(\fc)^2$ and the sign is multiplied by $\chi(-N)$. The key to finding the
mixed periods of $E$ is contained in the following conjecture that \dembele has distilled
from \cite{bertolini-darmon-green}.
\begin{conjecture}
If $\chi$ is a primitive quadratic character with conductor $\fc$ relatively prime to the
conductor of $E$, with $\chi(\varphi) = s'$ and
$\chi(1 - \varphi) = s$, (where $s, s' \in \{+, -\} = \{\pm1\}$), then
\[
    \Omega^{s,s'}_E = c_\chi \tau(\overline\chi) L(E, \chi, 1)\sqrt5,
\]
for some integer $c_\chi$, where $\tau(\chi)$ is the Gauss sum
\[
    \tau(\chi) = \sum_{\alpha \bmod \fc} \chi(\alpha) \exp\left(2 \pi i \Tr\left(\alpha/m\sqrt{5}\right) \right),
\]
with $m$ a totally positive generator of $\fc$.
\end{conjecture}
\begin{remark}
  This statement has been specialized to $\Q(\sqrt 5)$, but not very
  much. In the more general case of a real quadractic field, $5$ is
  replaced by the discriminant, and $\vphi$ is replaced by a
  fundamental unit.  The Gauss sum is more innocuous than it
  seems. For odd conductor $\fc$ it is of size $\sqrt{\Norm(\fc)}$, while
  for an even conductor it is of size $\sqrt{2 \Norm(\fc) }$.  Its sign is
  a $4$-th root of unity, and whether it is real or imaginary can be
  deduced directly from the conjecture, as it matches with the sign of
  $\Omega^{s,s'}_E$. In particular, $\tau(\chi)$ is real when
  $\chi(-1) = 1$ and imaginary when $\chi(-1) = -1$, which is a
  condition on $\Norm(\fc) \bmod 4$, as $\chi(-1) \equiv \Norm(\fc) \pmod
  4$. This can all be deduced, for example, from
  \cite{boylan-skoruppa:hecke-gauss-sums}.

Also, note that \dembele writes this conjecture with an additional factor of $4\pi^2$;
this factor does not occur with the definition of $L(f, s)$ that we have given.

\end{remark}

\begin{remark}
Contained in this conjecture is the obstruction to carrying out the method described here when 
$\n$ is a square: because of the sign of the functional equation, if $\n$ is a square $L(E, \chi, 1)$
will be nonvanishing only when $\chi(\varphi) = \chi(1 - \varphi)$, so we can only obtain
information about either $\Omegamm$ and $\Omegapp$ or $\Omegamp$ and $\Omegapm$, and we need
three of these values to find $E$.
\end{remark}

With this conjecture in place, we can describe a method for guessing the mixed periods of
$E$.
Now, to proceed, we construct four lists of characters
up to some conductor bound $M$ (we are restricting to odd prime modulus here for simplicity,
as primitivity is ensured, but this is not necessary):
\[
    S^{s,s'} = \{ \chi \bmod \p : \chi(\phi) = s', \chi(1 - \phi) = s,
            (\p, \n) = 1, \Norm(\p) < M, \chi(-N) = \epsilon_E\},
\]
Here $s, s' \in \{+, -\} = \pm 1$ again. We will consider these lists to be ordered by the
norm conductors of the characters in increasing order, and index their elements as
$\chi^{s,s'}_0, \chi^{s,s'}_1, \chi^{s,s'}_2, \ldots$. For each character we compute the central
value of the twisted $L$-function to get four new lists
\[
    \mathcal{L}^{s,s'} = \{ i^{ss'}\sqrt{5 \Norm(\p) } L(E,\chi,1), \chi \in S^{s,s'}\} =
        \{\mathcal{L}^{s,s'}_0, \mathcal{L}^{s,s'}_1, \ldots\}.
\]
These numbers should now all be integer multiples of the mixed periods, so to get an idea
of which integer multiples they might be, we compute each of the ratios 
\[
    \frac{\mathcal{L}^{s,s'}_0}{\mathcal{L}^{s,s'}_k} = \frac{c_{\chi^{s,s'}_0}}{c_{\chi^{s,s'}_k}} \in \Q, k = 1, 2, \ldots,
\]
attempt to recognize these as rational number, and 
choose as an initial guess
\[
    \Omega^{ss'}_{E, \mathrm{guess}} = \mathcal{L}^{s,s'}_0\left(\mathrm{lcm}\left\{ \mathrm{numerator}\left(\frac{\mathcal{L}^{s,s'}_0}{\mathcal{L}^{s,s'}_k}\right), k = 1,2, \ldots \right\}\right)^{-1}.
\]

\subsubsection{An example}
We give now an example for an elliptic curve which we were able to find only by using
this method. At level $\n = (-38\varphi + 26)$ we found a newform $f$, computed
\begin{multline*}
    a_{(2)}(f) = -1,\  a_{(-2\varphi + 1)}(f) = 1,\  a_{(3)}(f) = -1, \\ 
    a_{(-3\varphi + 1)}(f) = -1,\  a_{(-3*a + 2)}(f) = -6, \cdots, a_{(200\varphi - 101)}(f) = 168
\end{multline*}
and determined, by examining the $L$-function, that the sign of the functional equation
should be $-1$. (In fact, we do not really need to know the sign of the functional eqaution,
as we would quickly determine that $+1$ is wrong when attempting to find the mixed periods.)
Computing the sets of characters described above, and choosing the first $3$ of each, we
have
\begin{multline*}
S^{--} = \{\chi_{(\varphi + 6)}, \chi_{(7)}, \chi_{(7\varphi - 4)}\}, \ \ 
    S^{-+} = \{\chi_{(-3\varphi + 1)}, \chi_{(5 \varphi - 2)}, \chi_{(\varphi - 9)}\} \\
S^{+-} = \{\chi_{(-4\varphi + 3)}, \chi_{(5\varphi - 3)}, \chi_{(-2\varphi + 13)} \} \ \ 
    S^{++} = \{\chi_{(\varphi + 9)}, \chi_{(9\varphi - 5)}, \chi_{(\varphi + 13)} \}.
\end{multline*}

By using the $5133$ $a_{\p}(f)$ above as input to Rubinstein's lcalc, we compute the
lists of approximate values
\[
\begin{split}
\mathcal{L}^{--} &= \{-33.5784397862407, -3.73093775400387, -18.6546887691646 \} \\ 
\mathcal{L}^{-+} &= \{18.2648617736017i, 32.8767511924831i, 3.65297235421633i \} \\
\mathcal{L}^{+-} &= \{41.4805656925342i, 8.29611313850694i, 41.4805677827298i\} \\
\mathcal{L}^{++} &= \{32.4909970742969, 162.454985515474, 162.454973589303\}.
\end{split}
\]
Note that lcalc will warn us that we do not have enough coefficients to obtain good
accuracy, and we make no claim as far as the accuracy of these values is concerned.
Hoping that the ends will justify the means, we proceed forward.

Dividing each list by the first entry, and recognizing the quotients as rational
numbers, we get the lists
\[
\begin{split}
       \{1.000, 9.00000000005519, 1.80000000009351\}  &\approx \{1, 9, 9/5\}\\
       \{1.000, 0.555555555555555, 5.00000000068986 \} &\approx \{1, 5/9, 5\}\\
       \{1.000, 4.99999999999994, 0.999999949610245 \}  &\approx \{1, 5, 1\} \\ 
       \{1.000, 0.199999999822733, 0.200000014505165 \} &\approx \{1, 1/5, 1/5\},
\end{split}
\]
which may give an indication of the accuracy of our values. We now proceed with
the guesses
\begin{align*}
\phantom{MMMMM} \OmegammEguess &\approx -33.5784397862407/9 &\approx&\  -3.73093775402141 \phantom{MMMMM} \\
    \OmegampEguess &\approx 18.2648617736017i/5 &\approx&\  3.65297235472034i \\
    \OmegapmEguess &\approx 41.4805656925342i/5 &\approx&\  8.29611313850683i \\
    \OmegappEguess &\approx 32.4909970742969    &=&\  32.4909970742969.
\end{align*}
These cannot possibly be all correct, as $\Omegamm_E \Omegapp_E = \Omegamp_E \Omegapm_E$.
Still, we can choose any three and get a reasonable guess, and in fact we may choose all
possible triples, dividing some of the guesses by small rational numbers, and choosing the
fourth guess to be consistent with the first three; we build a list of possible embeddings
of $j(E)$, which will contain the possibility $\sigma_1(j(E)) \approx 1.365554233954 \times 10^{12}$,
$\sigma_2(j(E)) \approx 221270.95861123$, which is a possibility if 
\[\Omegamp_E = \OmegampEguess,\ \ 
\Omegapm_E = \OmegapmEguess,\ \ \Omegamp_E = \frac{\OmegampEguess}{2},\ \ \Omegapp_E = \frac{\OmegappEguess}{8}.\]
Cycling through many discriminants, we will eventually try
\[
    \Deltaguess = \varphi \cdot 2^5 \cdot (19\varphi - 13),
\]
which leads us to the guess
\begin{align*}
    \sigma_1(\cfourguess) &= (\sigma_1(j(E)) \sigma_1(\Deltaguess))^{1/3} \approx 107850.372979378 \\
    \sigma_2(\cfourguess) &= (\sigma_2(j(E)) \sigma_2(\Deltaguess))^{1/3} \approx 476.625892034286.
\end{align*}
We have enough precision to easily recognize this as
\[
    \cfourguess = \frac{108327 + 48019 \sqrt5}{2} = 48019\varphi + 30154,
\]
and
\[
    \sqrt{\cfourguess^3 - 1728\Deltaguess}
\]
does in fact have two square roots: $\pm(15835084\varphi + 9796985)$. We try both of them, and
the choice with the minus sign gives the curve
\[
y^2 + \varphi xy + \varphi y = x^3 + \left(\varphi - 1\right)x^2 + \left(-1001 \varphi - 628\right)x + \left(17899 \varphi + 11079\right),
\]
which has the correct conductor. We compute a few values of $a_\p$ for this curve, and it turns
out to be the one that we are looking for.

\section{Enumerating the curves in an isogeny class}\label{sec:isoclass}

Given an elliptic curve $E/F$, we wish to find representative
isomorphism classes for all elliptic curves $E'/F$ that are isogenous
to $E$ via an isogeny defined over $F$. The analogue of this problem
over $\Q$ has an algorithmic solution as explained in
\cite[\S3.8]{cremona:algs}; it relies on:
\begin{enumerate}
\item  Mazur's theorem \cite{mazur:rational} 
that if $\varphi:E\to E'$ is a $\Q$-rational isogeny of prime degree, 
then $\deg(\varphi)\leq 163$
\item Formulas of Velu \cite{velu:isogenies} 
that provide a way to explicitly enumerate all $p$-isogenies (if any) with domain $E$.  
Velu's formulas are valid for any number field, but so far there has not been an explicit
generalization of Mazur's theorem for any number field other than $\Q$. 
\end{enumerate}

\begin{remark}
  Assume the generalized Riemann hypothesis.  Then work of Larson-Vaintrob 
  from \cite{larson-vaintrob} implies that there is an effectively computable constant $C_F$ 
  such that any prime degree isogeny over $F$ has degree at most $C_F$.
\end{remark}

Since we are interested in individual isogeny classes, 
we can use the algorithm described in \cite{billerey:isog} that takes as
input a specific non-CM elliptic curve $E$ over a number field $K$, and
outputs a provably finite list of primes $p$ such that $E$ might have a
$p$-isogeny. The algorithm is particularly easy to implement in the
case when $K$ is a quadratic field, as explained in
\cite[\S2.3.4]{billerey:isog}.  Using this algorithm combined with
Velu's formulas, we were able to enumerate {\em all}
isomorphism classes of curves isogenous to the curves we found via the
methods of Section~\ref{sec:finding}, and thus divide our curves up into
isogeny classes.


\section{Related future projects}\label{sec:future}

It would be natural to extend the tables to the first known curve of
rank $3$, which may be the known one of norm conductor $163^2=26569$.
It would also be interesting to make a table in the style of
\cite{stein-watkins:ants5}, and compute analytic ranks of the large
number of curves that we would find; this would benefit from
Sutherland's smalljac program, which has very fast code for computing
$L$-series coefficients.  Some aspects of the tables could also be
generalized to modular abelian varieties $A_f$ attached to Hilbert
modular newforms with not-necessarily-rational Hecke eigenvalues; in
particular, we could enumerate the $A_f$ up to some norm conductor,
and numerically compute their analytic ranks.

\section{Tables}\label{sec:tables}

As explained in Sections~\ref{sec:finding} and \ref{sec:isoclass},
assuming Conjecture~\ref{conj:mod}, we found the complete list of
elliptic curves with norm conductor up to $1831$, which is the first
norm conductor of a rank $2$ curve over $F$.  Table~\ref{table:increasing-counts}
summarizes the number of curves we found.  Note that in these counts
we do not exclude conjugate curves, i.e., if $\sigma$ denotes the
nontrivial element of $\Gal(F/\Q)$, then we count $E$ and $E^{\sigma}$
separately if they are not isomorphic.  

\begin{center}
\begin{table}
\caption{Curves over $\Q(\sqrt{5})$ with norm conductor up to 1831\label{table:total-counts}}
\begin{tabular}{|c|c|c|c|}\hline
{\bf rank} & $\#\{${\bf isogeny classes }$\}$ &  $\#\{$ {\bf isom. classes }$\}$ &{\bf smallest $\Norm(\n_E)$} \\\hline
 $0$  & $745$  & $2172$ & $31$\\\hline
 $1$  & $667$  & $1190$ & $199$ \\\hline
 $2$  & $2$    & $2$    & $1831$ \\\hline
total & $1414$ & $3364$ & - \\\hline
\end{tabular}
\end{table}
\end{center}

\begin{center}
\begin{table}[h]
\caption{Counts of isogeny and isomorphism classes with bounded norm conductors and specified ranks\label{table:increasing-counts}}
\begin{tabular}{l|c|c|c|c|c|c|c|c|c|c|c|} \cline{2-11}
&\multicolumn{10}{|c|}{\textbf{norm conductor bound}} \\\cline{2-11}
& 200 & 400 & 600 & 800 & 1000 & 1200 & 1400 & 1600 & 1800 & 1831\\\hline\hline
\multicolumn{11}{|c|}{\textbf{number of isogeny classes }} \\\hline
\multicolumn{1}{|l|}{\textbf{rank 0}} & 62 & 151 & 246 & 334 & 395 & 492 & 574 & 669 & 729 & 745 \\\hline
\multicolumn{1}{|l|}{\textbf{rank 1}} & 2 & 32 & 94 & 172 & 237 & 321 & 411 & 531 & 655 & 667 \\\hline
\multicolumn{1}{|l|}{\textbf{rank 2}} & 0 & 0 & 0 & 0 & 0 & 0 & 0 & 0 & 0 & 2 \\\hline
\multicolumn{1}{|l|}{\textbf{total}} & 64 & 183 & 340 & 506 & 632 & 813 & 985 & 1200 & 1384 & 1414 \\\hline\hline
\multicolumn{11}{|c|}{\textbf{number of isomorphism classes}} \\\hline
\multicolumn{1}{|l|}{\textbf{rank 0}} & 257 & 580 & 827 & 1085 & 1247 & 1484 & 1731 & 1970 & 2128 & 2174 \\\hline
\multicolumn{1}{|l|}{\textbf{rank 1}} & 6 & 59 & 155 & 285 & 399 & 551 & 723 & 972 & 1178 & 1192 \\\hline
\multicolumn{1}{|l|}{\textbf{rank 2}} & 0 & 0 & 0 & 0 & 0 & 0 & 0 & 0 & 0 & 2 \\\hline
\multicolumn{1}{|l|}{\textbf{total}} & 263 & 639 & 982 & 1370 & 1646 & 2035 & 2454 & 2942 & 3306 & 3368 \\\hline
\end{tabular}
\end{table}
\end{center}

\begin{center}
\begin{table}[h]
\caption{Number of Isogeny classes of a given size\label{table:isogeny-sizes}}
\begin{tabular}{|l|c|c|c|c|c|c|c|c|c|}\hline
\textbf{size}                      & 1   & 2   & 3  & 4   & 6  & 8  & 10 & total \\\hline
\textbf{up to norm conductor 199}  & 2   & 21  & 3  & 20  & 8  & 9  & 1  & 64    \\\hline
\textbf{up to norm conductor 1831} & 498 & 530 & 36 & 243 & 66 & 38 & 3  & 1414  \\\hline
\end{tabular}
\end{table}
\end{center}

\begin{center}
\begin{table}[h]
\caption{Order of Torsion subgroups for curves with norm conductor up through 1831\label{table:tor-counts}}
\begin{tabular}{|l|c||l|c|}\hline
torsion structure    & $\#\{\text{ isom. classes }\}$ & torsion structure & $\#\{\text{ isom. classes }\}$ \\\hline
1                    & 296  & $\Z/8\Z$             & 21 \\\hline
$\Z/2\Z$             & 1453 & $\Z/4\Z\oplus\Z/2\Z$ & 51 \\\hline
$\Z/3\Z$             & 202  & $\Z/9\Z$             & 6  \\\hline
$\Z/4\Z$             & 243  & $\Z/10\Z$            & 12 \\\hline
$\Z/2\Z\oplus\Z/2\Z$ & 312  & $\Z/12\Z$            & 6  \\\hline
$\Z/5\Z$             & 56   & $\Z/6\Z\oplus\Z/2\Z$ & 11 \\\hline
$\Z/6\Z$             & 183  & $\Z/15\Z$            & 1  \\\hline
$\Z/7\Z$             & 13   & $\Z/8\Z\oplus\Z/2\Z$ & 2  \\\hline
\end{tabular}
\end{table}
\end{center}

\begin{center}
\begin{table}[h]
\caption{Conjectural $\#\Sha$ for curves with norm conductor up through 1831}\label{table:sha-counts}
\begin{tabular}{|l|c|c|c|c|c|c|}\hline
\#\Sha                       & 1    & 4  & 9  & 16 & 25 & 36 \\\hline
$\#\{\text{ isom. classes }\}$ & 3191 & 84 & 43 & 16 & 2 & 2 \\\hline
\end{tabular}
\end{table}
\end{center}

\begin{center}
\begin{table}[h]
\caption{First curve with given $\#\Sha$\label{table:sha}}
\begin{tabular}{|c | c | c |}
\hline 
$n$ &{\bf First curve $E$ having $|\Sha(E/F)| = n$ }& {\bf Norm of conductor of $E$} \\
\hline 
1 & $[1,\vphi{}+1,\vphi{},\vphi{},0]$ & 31 \\\hline 
4 & $[1, 1, 1, -110, -880]$ & 45 \\\hline 
9 & $[\vphi{}+1,-\vphi{},1,-54686\vphi{}-35336,$& \\
  & $\qquad\qquad\qquad -7490886\vphi{}-4653177]$ & 76 \\\hline 
16 & $[1,\vphi{},\vphi{}+1,-4976733\vphi{}-3075797,$&\\
   & $\qquad\qquad\qquad -6393196918\vphi{}-3951212998]$ & 45 \\\hline 
25 & $[0, -1, 1, -7820, -263580]$ & 121 \\\hline 
36 & $[1,-\vphi{}+1,\vphi{},1326667\vphi{}-2146665,$ & \\
   & $\qquad\qquad\qquad 880354255\vphi{}-1424443332]$ & 1580 \\
\hline
\end{tabular}
\end{table}
\end{center}

There are only 31 possible $j$-invariants for elliptic curves over $F$.  
For elliptic curves of norm conductor up to 1831, 9 of these $j$-invariants occur:
%[-12288000, 0, 287496, 1728, -85995*a - 52515, 16554983445*a +
%10231546590, 85995*a - 138510, -16554983445*a + 26786530035, 54000]

\begin{center}
\begin{table}[h]
\caption{CM $j$-invariants for curves of norm conductor up to 1831\label{table:CM-j-invariants}}
\begin{tabular}{|c|l|}\hline
{\bf label} &{\bf $j$-invariant} \\\hline
$j_1$ & $-12288000$ \\\hline
$j_2$ & $0$ \\\hline
$j_3$ & $ 287496$ \\\hline
$j_4$ & $1728$ \\\hline
$j_5$ & $-85995\vphi - 52515$ \\\hline
$j_6$ & $85995\vphi - 138510$ \\\hline
$j_7$ & $16554983445\vphi + 10231546590$ \\\hline
$j_8$ & $-16554983445\vphi + 26786530035$ \\\hline
$j_9$ & $-54000$ \\\hline
\end{tabular}
\end{table}
\end{center}

%this table ***INCLUDES*** all the missing curves.
\begin{center}
\begin{table}[h]
\caption{Curves over $\Q(\sqrt{5})$ with CM  and norm conductor up to 1831 \label{table:CM-curves}}
\begin{tabular}{|c|l|l|l|}\hline
{\bf $\n$} & {\bf a-invariants} & {\bf $j(E)$} &{\bf rank} \\\hline
 $9$ & $\left[1, -1, \vphi, -2\vphi, \vphi \right]$ & $j_5$ & 0 \\\hline %good
 %['81a', 'a1', 81, 9, [1, -1, a, -2*a, a]] -85995*a - 52515 0
 $9$ & $\left[1, -1, \vphi + 1, \vphi - 2, -2\vphi + 1\right]$ & $j_6$ & 0 \\\hline  %good
 %['81a', 'a2', 81, 9, [1, -1, a + 1, a - 2, -2*a + 1]] 85995*a - 138510 0
$9$ & $\left[\vphi, -\vphi - 1, \vphi, -128\vphi - 80, 921\vphi + 569\right]$ & $j_7$ & 0 \\\hline %good
%['81a', 'a3', 81, 9, [a, -a - 1, a, -128*a - 80, 921*a + 569]] 16554983445*a + 10231546590 0
$9$ & $\left[\vphi + 1, 1, 0, 129\vphi - 207, -793\vphi + 1283\right]$ & $j_8$ & 0\\\hline %good
%['81a', 'a4', 81, 9, [a + 1, 1, 0, 129*a - 207, -793*a + 1283]] -16554983445*a + 26786530035 0
$9$ & $\left[\vphi, -\vphi - 1, \vphi + 1, -1153\vphi - 719, -21854\vphi - 13501\right]$ & $j_7$& 0 \\\hline %good
%['81a', 'a5', 81, 9, [a, -a - 1, a + 1, -1153*a - 719, -21854*a - 13501]] 16554983445*a + 10231546590 0
$9$ & $\left[\vphi + 1, 1, 1, 1153\vphi - 1871, 23006\vphi - 37225\right]$ & $j_8$ & 0 \\\hline %good
%['81a', 'a6', 81, 9, [a + 1, 1, 1, 1153*a - 1871, 23006*a - 37225]] -16554983445*a + 26786530035 0
$9$ & $\left[1, -1, \vphi, 13\vphi - 15, 20\vphi - 26\right]$ & $j_6$& 0 \\\hline %good
%['81a', 'a7', 81, 9, [1, -1, a, 13*a - 15, 20*a - 26]] 85995*a - 138510 0
$9$ & $\left[1, -1, \vphi + 1, -14\vphi - 2, -21\vphi - 6\right]$ & $j_5$ & 0 \\\hline %good
%['81a', 'a8', 81, 9, [1, -1, a + 1, -14*a - 2, -21*a - 6]] -85995*a - 52515 0
$27$ & $\left[0, 0, 1, -30, 63\right]$ & $j_1$ & 1 \\\hline %good
%['729a', 'a1', 729, 27, [0, 0, 1, -30, 63]] -12288000 1
 $27$ & $\left[0, 0, 1, 0, 0\right]$ & $j_2$ & 1 \\\hline %good
 %['729a', 'a2', 729, 27, [0, 0, 1, 0, 0]] 0 1
$27$ & $\left[0, 0, 1, 0, -7\right]$ & $j_2$ & 1 \\\hline %good
%['729a', 'a3', 729, 27, [0, 0, 1, 0, -7]] 0 1
$27$ & $\left[0, 0, 1, -270, -1708\right]$ & $j_1$ & 1\\\hline %good
%['729a', 'a4', 729, 27, [0, 0, 1, -270, -1708]] -12288000 1
$27$ & $\left[0, 0, \vphi + 1, 0, -\vphi \right]$ & $j_2$& 0 \\\hline %good
%['729a', 'd1', 729, 27, [0, 0, a + 1, 0, -a]] 0 0
$27$ & $\left[0, 0, \vphi + 1, 0, 6\vphi - 145\right]$ & $j_2$ & 0\\\hline %good
%['729a', 'd2', 729, 27, [0, 0, a + 1, 0, 6*a - 14]] 0 0
$27$ & $\left[0, 0, \vphi, 0, -7\vphi - 7\right]$ & $j_2$ & 0 \\\hline %good
%[729, 27, 0, 0, [0, 0, a, 0, -7*a - 7]]
$27$ & $\left[0, 0, \vphi, 0, 0\right]$ & $j_2$ & 0 \\\hline %good
%[729, 27, 0, 0, [0, 0, a, 0, 0]]
$32$ & $\left[0, 0, 0, -11, 14\right]$ & $j_3$& 0 \\\hline %good
%['1024a', 'a1', 1024, 32, [0, 0, 0, -11, 14]] 287496 0
$32$ & $\left[0, 0, 0, -1, 0\right]$ & $j_4$ & 0 \\\hline %good
%['1024a', 'a2', 1024, 32, [0, 0, 0, -1, 0]] 1728 0
$32$ & $\left[0, 0, 0, -11, -14\right]$ & $j_3$ & 0 \\\hline %good
%['1024a', 'a3', 1024, 32, [0, 0, 0, -11, -14]] 287496 0
 $32$ & $\left[0, 0, 0, 4, 0\right]$ & $j_4$ & 0 \\\hline %good
 %['1024a', 'a4', 1024, 32, [0, 0, 0, 4, 0]] 1728 0
$32$ & $\left[0, 0, 0, -\vphi - 1, 0\right]$ & $j_4$ & 0 \\\hline %good
%['1024a', 'j1', 1024, 32, [0, 0, 0, -a - 1, 0]] 1728 0
$32$ & $\left[0, 0, 0, -11\vphi - 11, -28\vphi - 14\right]$ & $j_3$ & 0\\\hline %good
%['1024a', 'j2', 1024, 32, [0, 0, 0, -11*a - 11, -28*a - 14]] 287496 0
$32$ & $\left[0, 0, 0, -11\vphi - 11, 28\vphi + 14\right]$ & $j_3$& 0 \\\hline %good
%['1024a', 'j3', 1024, 32, [0, 0, 0, -11*a - 11, 28*a + 14]] 287496 0
$32$ & $\left[0, 0, 0, 4\vphi + 4, 0\right]$ & $j_4$ & 0\\\hline %good
%['1024a', 'j4', 1024, 32, [0, 0, 0, 4*a + 4, 0]] 1728 0
$36$ & $\left[0, 0, 0, -15, 2\right]$ & $j_9$& 0 \\\hline %good
%['1296a', 'b1', 1296, 36, [0, 0, 0, -15, 22]] 54000 0
$36$ & $\left[0, 0, 0, 0, 1\right]$ & $j_2$ & 0 \\\hline %good
%['1296a', 'b2', 1296, 36, [0, 0, 0, 0, 1]] 0 0
$36$ & $\left[0, 0, 0, -135, -594\right]$ & $j_9$ & 0 \\\hline %good
%['1296a', 'b3', 1296, 36, [0, 0, 0, -135, -594]] 54000 0
$36$ & $\left[0, 0, 0, 0, -27\right]$ & $j_2$ & 0 \\\hline %good
%['1296a', 'b4', 1296, 36, [0, 0, 0, 0, -27]] 0 0
\end{tabular}
\end{table}
\end{center}

\subsection{CM elliptic curves over $F$}\label{sec:cm}

There are $31$ distinct $\Qbar$-isomorphism classes of CM elliptic
curves defined over $F$, more than for any other
quadratic field.

\begin{proposition}
The field $F$ has more isomorphism classes of CM
elliptic curves than any other quadratic field.
\end{proposition}
\begin{proof}
Let $K$ be a quadratic extension of $\Q$.
Let $H_D$ denote the Hilbert class polynomial of the CM order
$\cO_D$ of discriminant $D$, so $H_D\in \Q[X]$ is the minimal
polynomial of the $j$-invariant $j_D$ of any elliptic curve $E=E_D$ with
CM by $\cO_D$.   Since $K$ is Galois, we have $j_D \in K$ if and only if
$H_D$ is either linear or quadratic and has both roots in $K$.
The $D$ for which $H_D$ is linear are the thirteen values
$-3, -4, -7, -8, -11, -12, -16, -19, -27, -28, -43, -67, -163$.
According to \cite{cremona:abvar}, the
$D$ for which $H_D$ is quadratic are the following $29$
discriminants
\begin{align*}
&-15, -20, -24, -32, -35, -36, -40, -48, -51, -52, -60, \\
&-64, -72, -75, -88, -91, -99, -100, -112, -115, -123, \\
&-147, -148, -187, -232, -235, -267, -403, -427
\end{align*}
%z = list(reversed(sorted([D*f^2 for D,f in cm_orders(2)]))); z 
%len(z)
 
By computing discriminants of these Hilbert class polynomials,
we obtain the following table:
% w = {}
% for D, f in cm_orders(2):
%     m = hilbert_class_polynomial(D*f^2).discriminant().squarefree_part()
%     print D, m
%     if not w.has_key(m):
%         w[m] = []
%     w[m].append(D)
%     w[m] = list(reversed(sorted(w[m])))
% print '-'*70
% print w

% for F, Ds in sorted(w.iteritems()):
%     print "\\Q(\\sqrt{%s}) & $%s$ \\\\\\hline"%(F,
%                   ','.join([str(z) for z in Ds]))

\begin{center}
\begin{tabular}{|c|l|}\hline
Field & $D$ so $H_D$ has roots in field\\\hline
$\Q(\sqrt{2})$ & $-24,-32,-64,-88$ \\\hline
$\Q(\sqrt{3})$ & $-36,-48$ \\\hline
$\Q(\sqrt{5})$ & $-15,-20,-35,-40,-60,$ \\
             & $-75,-100,-115,-235$\\\hline
$\Q(\sqrt{6})$ & $-72$ \\\hline
$\Q(\sqrt{7})$ & $-112$ \\\hline
$\Q(\sqrt{13})$ & $-52,-91,-403$ \\\hline
$\Q(\sqrt{17})$ & $-51,-187$ \\\hline
\end{tabular}
\begin{tabular}{|c|l|}\hline
Field & $D$ so $H_D$ has roots in field\\\hline
$\Q(\sqrt{21})$ & $-147$ \\\hline
$\Q(\sqrt{29})$ & $-232$ \\\hline
$\Q(\sqrt{33})$ & $-99$ \\\hline
$\Q(\sqrt{37})$ & $-148$ \\\hline
$\Q(\sqrt{41})$ & $-123$ \\\hline
$\Q(\sqrt{61})$ & $-427$ \\\hline
$\Q(\sqrt{89})$ & $-267$ \\\hline
& \\\hline
\end{tabular}
\end{center}
The claim follows because the $\Q(\sqrt{5})$ row is largest,
containing $9$ entries.  There are thus $31 = 2\cdot 9 + 13$ distinct CM $j$-invariants
in $\Q(\sqrt{5})$.

\end{proof}




\bibliographystyle{amsalpha} 
\bibliography{biblio}
\end{document}
